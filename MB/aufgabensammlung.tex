% -----------------------------------------------------------------
% Einstellungen f�r Sammlungen von Aufgaben
% Verwendet die Einstellung in �bungsaufgaben.tex
% Die Ausgabe der Aufgabennummer wird durch
% eine Box erg�nzt, die den Dateinamen anzeigt.
%
% Zwei weitere Variable werden definiert,
% f�r die Indizierung von Klausuraufgaben und
% die Verwendung von Aufgaben in �bungsbl�ttern.
% ---------------------------------------------------------------
\typeout{Einstellungen f�r Aufgabensammlungen}
\typeout{     (C) Manfred Brill}
\typeout{     Version 1.2 Juli 2019}
% Variable f�r die �bungsblatt-Nummer
\newcommand{\theBlatt}{Aufgabensammlung}
% Die Einstellung f�r �bungsbl�tter laden
% Seit Update Dateinamen ohne deutsche Umlaute
% ----------------------------------------
% Einstellung f�r Mathematik-�bungsbl�tter
% ----------------------------------------
\typeout{�bungsbl�tter in Mathematik-Veranstaltungen}
\typeout{     (C) Manfred Brill}
\typeout{     Version 1.5 April 2023}
% Voraussetzungen
% Struktur der Aufgaben und L�sungen wie erwartet.
% Lokal gibt es eine Datei stammdaten, mit Definitionen
% f�r das Semester, Studiengang etc.
\usepackage{mbPDF}
\usepackage{mbmath}
\usepackage{textcomp}
% Array-Paket f�r mehr Kontrolle der Tabellen
\usepackage{array}
% ifthen f�r Ein- und Ausblenden der L�sungen.
\usepackage{ifthen}
% Hierarchien mit dirtree
\usepackage{dirtree}
% Header f�r LV-Variablen
% Variablen f�r das Semester, die Vorlesung ...
\newcommand{\theSemester}{Wintersemester~2021/22}
% Variable f�r die Vorlesung
\newcommand{\theClass}{Styles and More}
% Variable f�r den Studiengang
\newcommand{\theCourse}{Dokumentation der Datei setup.tex}
% Variable f�r den Hochschul-Namen
\newcommand{\theSchool}{Hochschule~Kaiserslautern}
% Variable f�r den Dozenten
\newcommand{\theTeacher}{Manfred~Brill}

%
% Verzeichnisse f�r �bungsaufgaben, Bitmaps, ...
%

% Verzeichnis f�r die �bungsaufgaben und Musterl�sungen
\newcommand{\exercisePath}{./Uebungsaufgaben/}
% Verzeichnis f�r  Bilder
\newcommand{\imagePath}{./images}

%
% Abbildungen f�r die Titelseite oder spezielle Markierungen
%
% Default-Titelbild der Lehrveranstaltung
\newcommand{\titleImage}{\imagePath/tugboat}
% Abbildung f�r die Angabe von Begleittexten
\newcommand{\lesenImage}{\imagePath/Misc/buchicon}
% Abbildung f�r die Angabe von vertiefenden Texten
\newcommand{\vertiefenImage}{\imagePath/Misc/reading}
% Abbildung f�r eine Marginalie zum Praxisbezug
\newcommand{\praxisImage}{\imagePath/Misc/gabel}
% Literatur-Stil
\bibliographystyle{geralpha}
% Formeln nur lokal nummerieren. Der Counter wird bei jeder Aufgabe zur�ckgesetzt!
\renewcommand{\theequation}{\arabic{equation}}
%
\raggedbottom
\setlength{\parskip}{2.0ex}
\setlength{\parindent}{0.0cm}
% Verhindert Schusterjungen und Hurenkinder
\clubpenalty = 10000
\widowpenalty = 10000
\displaywidowpenalty = 10000
% Kopfzeilen mit KoMaScript
\usepackage[automark, headsepline]{scrlayer-scrpage}
%
\pagestyle{scrheadings}
\clearscrheadfoot
% Thema des �bungsblatts in die Kopfzeile
\ihead[]{\theBlatt{}}
\ohead[]{\pagemark}
\chead{}
\pagestyle{scrheadings}
% Farben
% ---------------------------------------------------------------------------------------
% Farben, f�r Folien und andere Dinge.
% Version vom Juli 2015 mit neuer Fachbereichsfarbe
%
% Letzte �nderung: 1.9.2015
%
% Um diese Datei zu verwenden muss Sie in texmf-local/MB kopiert werden
% und TeX muss aktualisiert werden!
% ---------------------------------------------------------------------------------------
\definecolor{light}{gray}{.75}
\definecolor{LightGray}{rgb}{0.24,0.24,0.24}
%   Farben rot gr�n blau f�r Koordinatenachsen
\definecolor{wred}{rgb}{0.8, 0.0, 0.0}
\definecolor{wgreen}{rgb}{0.0, 0.8, 0.0}
\definecolor{wblue}{rgb}{0.0, 0.0, 0.8}
\definecolor{wyellow}{rgb}{1.0, 1.0, 0.0}
\definecolor{wcyan}{rgb}{0.0, 1.0, 1.0}
% Fachbereichsfarbe neu
\definecolor{colorDepartment}{rgb}{0.16, 0.71, 0.86}
% Listings
\definecolor{lstback}{gray}{0.85}
%

% Variablen f�r Namen, Software, ...
% -------------------------------------------------------------------
% variablen.tex
% Variablen f�r Werkzeuge, Begriffe, Firmen
% Stand: Juni 2017
% Liegt zentral in texmf-local/tex/latex/MB
% -------------------------------------------------------------------
% Programmiersprachen
\newcommand{\java}{\texttt{Java}}
\newcommand{\cpp}{\texttt{C++}}
\newcommand{\cl}{\texttt{C}}
\newcommand{\csharp}{\texttt{C\#}}
\newcommand{\php}{\texttt{PHP}}
\newcommand{\py}{\texttt{Python}}
\newcommand{\perl}{\texttt{Perl}}
\newcommand{\fc}{\texttt{Fortran}}
\newcommand{\obc}{\texttt{Objective C}}
\newcommand{\js}{\texttt{JavaScript}}
\newcommand{\omp}{\texttt{OpenMP}}
\newcommand{\mpi}{\texttt{MPI}}
\newcommand{\ocl}{\texttt{OpenCL}}
\newcommand{\cuda}{\texttt{CUDA}}
% Datei-Formate
\newcommand{\html}{\texttt{HTML}}
\newcommand{\xml}{\texttt{XML}}
\newcommand{\json}{\texttt{JSON}}
\newcommand{\rtf}{\texttt{RTF}}
\newcommand{\pdf}{\texttt{PDF}}
%
\newcommand{\sccs}{\texttt{SCCS}}
\newcommand{\rcs}{\texttt{RCS}}
\newcommand{\cvs}{\texttt{CVS}}
\newcommand{\svn}{\texttt{SVN}}
\newcommand{\subversion}{\texttt{Subversion}}
\newcommand{\git}{\texttt{Git}}
\newcommand{\github}{\texttt{GitHub}}
\newcommand{\githubDesktop}{\texttt{GitHub Desktop}}
\newcommand{\gitkraken}{\texttt{GitKraken}}
\newcommand{\gittortoise}{\texttt{TortoiseGit}}
\newcommand{\gittree}{\texttt{Sourcetree}}
\newcommand{\bitbucket}{\texttt{BitBucket}}
\newcommand{\mercurial}{\texttt{Mercurial}}
\newcommand{\pf}{\texttt{Perforce}}
\newcommand{\pfv}{\texttt{P4V}}
\newcommand{\pfc}{\texttt{p4}}
\newcommand{\pfmerge}{\texttt{P4Merge}}
\newcommand{\pfsand}{\texttt{P4.Sandbox}}

\newcommand{\vi}{\texttt{vi}}
\newcommand{\emacs}{\texttt{Emacs}}
%
\newcommand{\javadoc}{\texttt{Javadoc}}
\newcommand{\xmldoc}{\texttt{XMLdoc}}
\newcommand{\sandcastle}{\texttt{Sandcastle}}
\newcommand{\doxy}{\texttt{Doxygen}}
\newcommand{\doxyVersion}{\texttt{$1.8.7$}}
\newcommand{\doxywizard}{\texttt{Doxywizard}}
\newcommand{\graphviz}{\texttt{Graphviz}}
%
\newcommand{\logj}{\texttt{Log4j}}
\newcommand{\slfj}{\texttt{SLF4J}}
\newcommand{\logback}{\texttt{LOGBack}}
\newcommand{\jul}{\texttt{Java Logging API}}
\newcommand{\commonsLog}{\texttt{Apache Commons LoggingJ}}
\newcommand{\boostlog}{\texttt{Boost.Log}}
\newcommand{\glog}{\texttt{Google Logging Library}}
%
\newcommand{\junit}{\texttt{JUnit}}
\newcommand{\xunit}{\texttt{xUnit}}
\newcommand{\nunit}{\texttt{NUnit}}
\newcommand{\xunitnet}{\texttt{xUnit.net}}
\newcommand{\ctest}{\texttt{CTest}}
%
\newcommand{\ssh}{\texttt{ssh}}
\newcommand{\sftp}{\texttt{sftp}}
\newcommand{\make}{\texttt{make}}
\newcommand{\ant}{\texttt{Ant}}
\newcommand{\maven}{\texttt{Maven}}
\newcommand{\gradle}{\texttt{Gradle}}
\newcommand{\msbuild}{\texttt{MSBuild}}
\newcommand{\mstest}{\texttt{MSTest}}
\newcommand{\cmake}{\texttt{CMake}}
\newcommand{\jenkins}{\texttt{Jenkins}}
\newcommand{\ccontrol}{\texttt{cruisecontrol}}
\newcommand{\cdash}{\texttt{CDash}}
%
\newcommand{\eclipse}{\texttt{Eclipse}}
\newcommand{\vs}{\texttt{Visual Studio}}
\newcommand{\mono}{\texttt{Mono}}
\newcommand{\monodev}{\texttt{MonoDevelop}}
%
\newcommand{\openO}{\texttt{OpenOffice}}
\newcommand{\Qt}{\texttt{Qt}}
\newcommand{\firefox}{\texttt{Mozilla Firefox}}
\newcommand{\kde}{\texttt{KDE}}
%
\newcommand{\ms}{\texttt{Microsoft}}
\newcommand{\net}{\texttt{.NET}}
\newcommand{\sun}{\texttt{SUN}}
\newcommand{\ora}{\texttt{Oracle}}
\newcommand{\apache}{\texttt{Apache}}
\newcommand{\windows}{\texttt{Windows}}
\newcommand{\cyg}{\texttt{Cygwin}}
\newcommand{\linux}{\texttt{Linux}}
\newcommand{\unix}{\texttt{Unix}}
\newcommand{\osx}{\texttt{MacOS X}}
\newcommand{\android}{\texttt{Android}}
% VR Software
\newcommand{\unity}{\texttt{Unity}}
\newcommand{\verUnity}{\texttt{$2018.2$}}
\newcommand{\unreal}{\texttt{Unreal}}
\newcommand{\verUnreal}{\texttt{4}}
\newcommand{\juggler}{\texttt{VRJuggler}}
\newcommand{\cavelib}{\texttt{CAVELib}}
\newcommand{\gadgeteer}{\texttt{Gadgeteer}}
\newcommand{\vrpn}{\texttt{VRPN}}
\newcommand{\mvr}{\texttt{MiddleVR}}
\newcommand{\verMvr}{\texttt{$1.7.0.7$}}
% Computergrafik und Werkzeuge
\newcommand{\direct}{\texttt{Direct3D}}
\newcommand{\gl}{\texttt{OpenGL}}
\newcommand{\para}{\texttt{ParaView}}
\newcommand{\vtk}{\texttt{VTK}}
% R und anderes zur Datenanalyse
% \R ist schon definiert (Mathematik!)
\newcommand{\Rsoft}{\texttt{R}}
\newcommand{\Rstudio}{\texttt{RStudio}}
\newcommand{\verR}{\texttt{$3.3.3$}}
\newcommand{\mondrian}{\texttt{Mondrian}}
\newcommand{\tableau}{\texttt{Tableau}}
% LaTeX
\newcommand{\jabref}{\texttt{JabRef}}
\newcommand{\bibtex}{\textsc{Bib}\negthinspace\TeX}

%++++++++++++++++++++++++++++++++++++++++++++++++++++++++++++++++
% Abst�nde zwischen Caption und Bild/Tabelle
\setlength\abovecaptionskip          {0.4em}
\setlength\belowcaptionskip          {0.2em}
\renewcommand{\topfraction}{0.99}
% Font und Fettdruck f�r Tabelle/Abbildung (neu, jetzt mit komascript!
% MB, 28/07/2017
\addtokomafont{caption}{\small}
\setkomafont{captionlabel}{\bfseries}
% Einstellung f�r Gliederungs�berschriften mit KoMaScript
\RedeclareSectionCommands[
  beforeskip=-.5\baselineskip,
  afterskip=.25\baselineskip
]{section,subsection,subsubsection}
%
%listings
%   Hintergrundfarbe von Quellcode
\definecolor{codecolor}{rgb}{0.85,0.85,0.85}
\lstloadlanguages{[ANSI]C++}
\lstset{basicstyle = \ttfamily \small}
\lstset{backgroundcolor=\color{codecolor}}
\lstset{extendedchars=true} \lstset{showstringspaces = false}
% ++++++++++++++++++++++++++++++++++++++++++++++++++++++++++++++++
\lstloadlanguages{Java}
\lstset{basicstyle = \ttfamily \small}
\lstset{backgroundcolor=\color{codecolor}}
\lstset{extendedchars=true}
%
% Aufgabenmakros zum Einblenden von Musterl�sungen
%
% Schalter f�r das ein- und ausblenden der L�sungen
\newboolean{solutions}
%
\theoremstyle{break} % Zeilenumbruch bei Aufgaben�berschrift
\theorembodyfont{\normalfont}
\newtheorem{auftitle}{Aufgabe}
% Teilaufgaben alphabetisch nummerieren
\renewcommand{\labelenumi}{\alph{enumi})}
% Funktion f�r eine Aufgabe im Dokument
\newcommand{\aufgabentext}[1]{\auftitle\label{#1}\input{aufgaben/aufgabenstellungen/#1}}
% Funktion f�r eine Aufgabe im Dokument ohne L�sung
\newcommand{\offeneAufgabe}[1]{\setcounter{equation}{0}\filename{#1.tex}\aufgabentext{#1}}
% Counter f�r equation auf null; damit die Gleichungen immer f�r jede Aufgabe neu nummeriert werden!
\newcommand{\aufgabe}[1]{\setcounter{equation}{0}\aufgabentext{#1}\solaufgabe{#1}}
% Funktion f�r eine L�sung im Dokument
\newcommand{\solaufgabe}[1]%
{\ifthenelse{\boolean{solutions}}{\subsubsection*{L�sung}\input{aufgaben/loesungen/#1}}{}}
% Funktion f�r eine Aufgabe zur Nachbereitung
\newcommand{\nachbereitung}[1]{%
\ifthenelse{\boolean{solutions}}{}{\subsubsection*{Nachbereitung} \label{#1} \input{nachbereitung/#1}}
}
% Kommando f�r den Flattersatz bei nebeneinander liegenden
% Abbildungen
\newcommand{\flatter}{\setlength{\rightskip}{0pt plus 2cm}}
% Anteil der Grafiken h�her auf jeder Seite!
\renewcommand{\textfraction}{0.001}
%
% Schritte in einer Aufz�hlung, daf�r einen Z�hler (schritt) und die Umgebung
% schritte definieren.
\newcounter{schritt}
\newenvironment{schritte}%
{\begin{list}%
{Schritt \arabic{schritt}:}%
{\usecounter{schritt}\settowidth{\labelwidth}{Schritt 1:}%
\setlength{\leftmargin}{\labelwidth}\addtolength\leftmargin{\labelsep}%
\parsep0.0ex\partopsep-0.3ex\itemsep2pt\topsep0.0ex}}{\end{list}}
%
\newcommand{\algorithmus}[2]{%
\vspace{4pt}\fboxsep 1mm \framebox[155mm]
{\parbox{145mm}{{\textbf{#1}}\vspace{2pt}#2}}\vspace{4pt}}
%
%   Tip (in einer Box)
\newcommand{\tip}[1]{
\begin{quote}\fboxsep 3mm\framebox[140mm][c]{\parbox{130mm}{{\textbf{Tipp}:\\}#1}}\end{quote}}

% Darstellung von Punkten mit xpicture mit Hilfe von \pointmark
\renewcommand{\pointmark}{$\bullet$}
% Funktion, der die Koordinaten �bergeben werden und einen Punkt ausgibt
\newcommand{\drawPoint}[2]{\Put[c](#1, #2){\pointmark}}

\ihead{\headmark}
%
% Einblenden von Musterl�sungen
%
\newcommand{\filename}[1]{%
\ifthenelse{\boolean{solutions}}{\framebox[50mm]{\parbox{40mm}{\textbf{#1}}}\vspace{6pt}}{}
}
% aufgabe �berschreiben, Dateiname mit ausgeben
\renewcommand{\aufgabe}[1]{\setcounter{equation}{0}\filename{#1.tex}\aufgabentext{#1}\solaufgabe{#1}}

% Funktion f�r das Verzeichnis der in einer Klausur verwendeten Aufgaben
\newcommand{\klausur}[1]{\ref{#1}&\lstinline$#1$}
%
% Funktion f�r das Verzeichnis der in den �bungsbl�ttern verwendeten Aufgaben
\newcommand{\uebung}[1]{#1 (Seite \pageref{#1})}
%
% Datensatz-Texte und Daten in eigener Datei
% chapter- und sectionmark mit der Kurzbezeichnung
% Die Kurzbezeichnungen werden mit addcontentsline dem Inhaltsverzeichnis hinzugef�gt
\newcommand{\dataset}[3]{\newpage\hrule\textbf{\large #1}\addcontentsline{toc}{section}{#1}\hspace*{2cm}\textbf{\large #2}\vspace*{0.3cm}\hrule\label{#3}\input{\dataPath/datasets/#3}\sectionmark{#1}\chaptermark{#1}}
%

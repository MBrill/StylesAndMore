% -----------------------------------------------------------------
% Einstellungen f�r Sammlungen von Aufgaben
% Verwendet die Einstellung in �bungsaufgaben.tex
% Die Ausgabe der Aufgabennummer wird durch
% eine Box erg�nzt, die den Dateinamen anzeigt.
%
% Zwei weitere Variable werden definiert,
% f�r die Indizierung von Klausuraufgaben und
% die Verwendung von Aufgaben in �bungsbl�ttern.
% ---------------------------------------------------------------
\typeout{Einstellungen f�r Aufgabensammlungen}
\typeout{     (C) Manfred Brill}
\typeout{     Version 1.1 Februar 2018}
% Variable f�r die �bungsblatt-Nummer
\newcommand{\theBlatt}{Aufgabensammlung}
% Die Einstellung f�r �bungsbl�tter laden
\input{�bungsblatt}
\ihead{\headmark}
%
% Einblenden von Musterl�sungen
%
\newcommand{\filename}[1]{%
\ifthenelse{\boolean{solutions}}{\framebox[50mm]{\parbox{40mm}{\textbf{#1}}}\vspace{6pt}}{}
}
% aufgabe �berschreiben, Dateiname mit ausgeben
\renewcommand{\aufgabe}[1]{\setcounter{equation}{0}\filename{#1.tex}\aufgabentext{#1}\solaufgabe{#1}}

% Funktion f�r das Verzeichnis der in einer Klausur verwendeten Aufgaben
\newcommand{\klausur}[1]{\ref{#1}&\lstinline$#1$}
%
% Funktion f�r das Verzeichnis der in den �bungsbl�ttern verwendeten Aufgaben
\newcommand{\uebung}[1]{#1 (Seite \pageref{#1})}
%
% Datensatz-Texte und Daten in eigener Datei
% chapter- und sectionmark mit der Kurzbezeichnung
% Die Kurzbezeichnungen werden mit addcontentsline dem Inhaltsverzeichnis hinzugef�gt
\newcommand{\dataset}[3]{\newpage\hrule\textbf{\large #1}\addcontentsline{toc}{section}{#1}\hspace*{2cm}\textbf{\large #2}\vspace*{0.3cm}\hrule\label{#3}\input{datasets/#3}\sectionmark{#1}\chaptermark{#1}}
%

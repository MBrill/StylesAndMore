\chapter{Packages and Style Files}
\section{The setup file setup\_english.tex}
To make sure the header of the main \LaTeX{}-file is not too long
we put most of the settings in a file \datei{setup\_english}.
There we load all needed packages and do the settings.

The rest can be seen in the documentation of the corresponding german
version. We use an external file with definitions for class names, teacher names
and more.

Here you find the values used for this document:
\begin{lstlisting}{}
% Variablen f�r das Semester, die Vorlesung ...
\newcommand{\theSemester}{Summer~Term~2020}
% Variable f�r die Vorlesung
\newcommand{\theClass}{Styles and More}
% Variable f�r den Studiengang
\newcommand{\theCourse}{Documentation of Datei setup\_english.tex}
% Variable f�r den Namen der Hochschule
\newcommand{\theSchool}{University~of
    ~Applied~Sciences~Kaiserslautern}
% Variable f�r den Dozenten
\newcommand{\theTeacher}{Manfred~Brill}
%
% Verzeichnisse f�r �bungsaufgaben, Bitmaps, ...
%

% Verzeichnis f�r die �bungsaufgaben und Musterl�sungen
\newcommand{\exercisePath}{./Uebungsaufgaben/}
% Verzeichnis f�r  Bilder
\newcommand{\imagePath}{./images}

%
% Abbildungen f�r die Titelseite oder spezielle Markierungen
%
% Default-Titelbild der Lehrveranstaltung
\newcommand{\titleImage}{\imagePath/tugboat}
% Abbildung f�r die Angabe von Begleittexten
\newcommand{\lesenImage}{\imagePath/Misc/buchicon}
% Abbildung f�r die Angabe von vertiefenden Texten
\newcommand{\vertiefenImage}{\imagePath/Misc/reading}
% Abbildung f�r eine Marginalie zum Praxisbezug
\newcommand{\praxisImage}{\imagePath/Misc/gabel}
% Abbildung f�r ein Smiley
\newcommand{\smileImage}{\imagePath/Misc/smilie} 
\end{lstlisting}
%
\subsection{Used \LaTeX{}-Packages}
The file \datei{setup\_englisch.tex} use style files like the german version,
especially \datei{mbPDF\_english.sty} und \datei{mbmath.sty}. 
The big difference is the usage of \lstinline$babel$ with the option \lstinline$english$.
%
\subsection{Functions and Macros}
All other functions and macros can be used like in the german version. Just make sure
you use proper englisch.
%
\section{References}
This document uses the package \lstinline$bibgerm$ to reuse the german \lstinline$bib$-file
from the german documentation. Just get rid of this package by creating a proper english
\lstinline$bib$-file!
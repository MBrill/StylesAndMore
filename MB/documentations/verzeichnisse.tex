\chapter{Verzeichnisse und Index}
\section{Literaturverzeichnis}
Das \begriff{Literaturverzeichnis} wird mit Hilfe von \bibtex{} aufgebaut. \bibtex{} ist relativ alt und unhandlich, das macht sich h�ufig bemerkbar. Es ist geplant auf
\emph{BibLaTeX} umzusteigen.

Die einzelnen Eingabe-Dateien f�r die Quellen werden mit Hilfe von \jabref{}, einer
\java{}-Anwendung\randnotiz{JabRef}\index{JabRef}, verwaltet (\cite{jabref}). Vorteil dieser Anwendung ist, dass man in den Dateien suchen kann
und die Eintr�ge als Formular-Eintr�ge bearbeitet werden k�nnen. Das Literaturverzeichnis,
das wir auf Seite \pageref{literatur} finden wurde so in das Dokument eingef�gt:

\begin{lstlisting}{}
\cleardoublepage
\phantomsection
\addcontentsline{toc}{chapter}{Literaturverzeichnis}
\chaptermark{Literaturverzeichnis}
\sectionmark{Literatur}\label{literatur}
\bibliography{../../../BibTex/latex}
\end{lstlisting}
Interessant ist die Funktion \lstinline$phantomsection$, die daf�r sorgt, dass das Literaturverzeichnis
richtig behandelt wird und die Verweise in der Inhaltsangabe bei der \pdf{}-Anzeige auch
korrekt auf den Anfang des Literaturverzeichnisses verweisen.
%
\section{Index}
Der \begriff{Index} wird mit dem Paket \lstinline$imakeidx$ erzeugt. Hier ist wichtig, dass die deutschen Umlaute korrekt
dargestellt werden. Dies gelingt durch die Option \lstinline$-g$ f�r die Funktion \lstinline$makeindex$, die
w�hrend der Erstellung des \pdf{}-Dokuments aufgerufen wird. Damit diese Option korrekt arbeitet und
deutsche Umlaute wie gewohnt im Editor eingegeben werden k�nnen m�ssen die Quotes umgestellt werden. Dazu gibt
es die Datei \datei{german.ist}, die sinnvoller Weise in \datei{texmf-local/makeindex/german} gespeichert ist:\index{Index!deutsche Umlauge}
\begin{lstlisting}{}
% Voraussetzung: makeindex mit Option -g aufrufen!
quote '>'    % > ersetzt "
%
delim_0 "\\hspace{2ex} "
delim_1 "\\hspace{2ex} "
delim_2 "\\hspace{2ex} "
\end{lstlisting}
F�r den Index ist eingestellt, dass ein Eintrag im Inhaltsverzeichnis erstellt wird und dass der Index, der mit \lstinline$printindex$
ausgegeben wird, zweispaltig ist. In diesem Dokument wurde der Index so eingef�gt:
\begin{lstlisting}{}
% Index
\clearevenpage
\phantomsection
\small
\printindex
\normalsize
\end{lstlisting}

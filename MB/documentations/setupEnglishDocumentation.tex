% -----------------------------------------------------------------------
% Documentation of setup_english
% -----------------------------------------------------------------------
\documentclass[enabledeprecatedfontcommands,
     fontsize=12pt,
     open=right, a4paper,
     twoside, DIV=11,
     abstractoff,
     headsepline,
     numbers=noenddot,
     BCOR=15mm,
     headings=standardclasses,
     headings=big]{scrbook}
\KOMAoptions{cleardoublepage=empty}
%
% Header f�r Variablen  wie Vorlesung, Semester, Pfade zu Aufgaben, Bildern, ...
% Variablen f�r das Semester, die Vorlesung ...
\newcommand{\theSemester}{November~2023}
% Variable f�r die Vorlesung
\newcommand{\theClass}{Styles and More}
% Variable f�r den Studiengang
\newcommand{\theCourse}{Documentation of Datei setup\_english.tex}
% Variable f�r den Namen der Hochschule
\newcommand{\theSchool}{University~of~Applied~Sciences~Kaiserslautern}
% Variable f�r den Dozenten
\newcommand{\theTeacher}{Manfred~Brill}
%
% Verzeichnisse f�r �bungsaufgaben, Bitmaps, ...
%

% Verzeichnis f�r die �bungsaufgaben und Musterl�sungen
\newcommand{\exercisePath}{./Uebungsaufgaben/}
% Verzeichnis f�r  Bilder
\newcommand{\imagePath}{./images}

%
% Abbildungen f�r die Titelseite oder spezielle Markierungen
%
% Default-Titelbild der Lehrveranstaltung
\newcommand{\titleImage}{\imagePath/tugboat}
% Abbildung f�r die Angabe von Begleittexten
\newcommand{\lesenImage}{\imagePath/Misc/buchicon}
% Abbildung f�r die Angabe von vertiefenden Texten
\newcommand{\vertiefenImage}{\imagePath/Misc/reading}
% Abbildung f�r eine Marginalie zum Praxisbezug
\newcommand{\praxisImage}{\imagePath/Misc/gabel}
% Abbildung f�r ein Smiley
\newcommand{\smileImage}{\imagePath/Misc/smilie} 
% Einstellungen
\input{setup_english}
% Daten zur Lehrveranstaltung
% Datensatz-Texte und Daten in eigener Datei
\newcommand{\dataset}[2]{\textbf{\large Dataset #1}\vspace*{0.3cm}\hrule\label{#2}\input{\exercisePath/datasets/#2}} 
%
% bibgerm, da deutsche Eintr�ge in bib-Datei.
% Falls hier englische Eintr�ge verwendet werden
% kann diese Zeile kommentiert werden.
\usepackage{bibgerm}
%
% Hyperref-Optionen f�r PDF-Files
% Dieses Paket muss als letztes eingebunden werden. Deshalb
% wird es hier "lokal" und nicht in setup.tex aufgef�hrt!
\usepackage[breaklinks]{hyperref}
\usepackage{breakurl}

\hypersetup{
pdftitle = {Template for documents in englisch language at the University of Applied Sciences Kaiserslautern},
pdfauthor = {Manfred Brill},
pdfsubject = {Template, LaTeX, Department of Computer Science and Microsystems Technology},
pdfkeywords = {Department of Computer Science and Microsystems Technology, University of Applied Sciences Kaiserslautern},
pdfdisplaydoctitle = true,
pdfpagemode = UseThumbs,
colorlinks = false,
linkcolor = green,
linkbordercolor = {0 1 0},
pdfpagelayout = {SinglePage}
}

\urlstyle{same}

%
% Schalter f�r das Ein- und Ausblenden der L�sungen
\setboolean{solutions}{true}
%
% Beginn Dokument
%
\begin{document}
% Es gibt eine Titelseite mit und ohne Bild. Beide Varianten kann man hier testen,
% durch Verschieben des Kommentarzeichens.
%\titelseite{\LaTeX{}-Texte an der Hochschule Kaiserslautern}
\titelseiteMitBild[\imagePath/tugboat]{\LaTeX{}-Texte an der Hochschule Kaiserslautern}
%
% Inhaltsverzeichnis
%
\tableofcontents
\clearevenpage
\pagenumbering{arabic}
%
%
%
\chapter{Introduction}
This document is a short introduction for a template we use for documents in classes and seminars
at the \theSchool{}. The basis for this template is a german version, and this document is based
on a paper on documentation of online texts in Microsoft Word (\cite{zfh_11}).

We suppose you have a working knowledge in \LaTeX{}. We still use
BibTeX for the references, which might cause problems, especially if you have german "`Umlaute"'
in the entries. We plan on switching to \emph{biber} in the future.
More on \LaTeX{} can be found in\cite{mittelbach_05} und \cite{kopka_02} (sorry for the german references \includegraphics[width=0.4cm]{\smileImage}).

Tests are done on Windows 10 and MiKTeX 2.9. The german versions are also tested on a Debian distribution.
It is known that students also use this files on a Mac.
%
%
%
\chapter{Packages and Style Files}
\section{The setup file setup\_english.tex}
To make sure the header of the main \LaTeX{}-file is not too long
we put most of the settings in a file \datei{setup\_english}.
There we load all needed packages and do the settings.

The rest can be seen in the documentation of the corresponding german
version. We use an external file with definitions for class names, teacher names
and more.

Here you find the values used for this document:
\begin{lstlisting}{}
% Variablen f�r das Semester, die Vorlesung ...
\newcommand{\theSemester}{Summer~Term~2020}
% Variable f�r die Vorlesung
\newcommand{\theClass}{Styles and More}
% Variable f�r den Studiengang
\newcommand{\theCourse}{Documentation of Datei setup\_english.tex}
% Variable f�r den Namen der Hochschule
\newcommand{\theSchool}{University~of
    ~Applied~Sciences~Kaiserslautern}
% Variable f�r den Dozenten
\newcommand{\theTeacher}{Manfred~Brill}
%
% Verzeichnisse f�r �bungsaufgaben, Bitmaps, ...
%

% Verzeichnis f�r die �bungsaufgaben und Musterl�sungen
\newcommand{\exercisePath}{./Uebungsaufgaben/}
% Verzeichnis f�r  Bilder
\newcommand{\imagePath}{./images}

%
% Abbildungen f�r die Titelseite oder spezielle Markierungen
%
% Default-Titelbild der Lehrveranstaltung
\newcommand{\titleImage}{\imagePath/tugboat}
% Abbildung f�r die Angabe von Begleittexten
\newcommand{\lesenImage}{\imagePath/Misc/buchicon}
% Abbildung f�r die Angabe von vertiefenden Texten
\newcommand{\vertiefenImage}{\imagePath/Misc/reading}
% Abbildung f�r eine Marginalie zum Praxisbezug
\newcommand{\praxisImage}{\imagePath/Misc/gabel}
% Abbildung f�r ein Smiley
\newcommand{\smileImage}{\imagePath/Misc/smilie} 
\end{lstlisting}
%
\subsection{Used \LaTeX{}-Packages}
The file \datei{setup\_englisch.tex} use style files like the german version,
especially \datei{mbPDF\_english.sty} und \datei{mbmath.sty}. 
The big difference is the usage of \lstinline$babel$ with the option \lstinline$english$.
%
\subsection{Functions and Macros}
All other functions and macros can be used like in the german version. Just make sure
you use proper englisch.
%
\section{References}
This document uses the package \lstinline$bibgerm$ to reuse the german \lstinline$bib$-file
from the german documentation. Just get rid of this package by creating a proper english
\lstinline$bib$-file!
%
%
% Literatur
%
\cleardoublepage
\phantomsection
\addcontentsline{toc}{chapter}{References}
\chaptermark{References}
\sectionmark{References}\label{literatur}
\bibliography{latex}
%
% Anhang
%
\appendix
%
% Index
\clearevenpage
\phantomsection
\small
\printindex
\normalsize
\end{document}

%
% Datensatz Berufs-Gruppen und sportliche Bet�tigung aus Bamberg/Baur
%
\subsection*{Beschreibung}\index{Beruf und Sport}
In einer Befragung wurden $1\:000$ berufst�tige Personen nach der Berufsgruppe
und der sportlichen Bet�tigung gefragt. Quelle ist \cite{bamberg_93}.
%
\subsection*{Format}
Die Kontingenztabelle der Ergebnisse ist in Tabelle \ref{berufsport:urliste} zusammengefasst.

\begin{table}[ht]
\begin{center}
\caption{\label{berufsport:urliste}Die Ergebnisse f�r den Datensatz "`Beruf und Sport"'}
\begin{tabular}{c|c|c|c|c}\hline
                                &\multicolumn{3}{c|}{\small{Sportliche Bet�tigung}}&\\\hline
\small{Berufsgruppe}&Nie&Gelegentlich&Regelm��ig&Randh�ufigkeit\\\hline
Arbeiter&240&120&70&430\\
Angestellter&160&90&90&340\\
Beamter&30&30&30&90\\
Landwirt&37&7&6&50\\
Freiberuflich&40&32&18&90\\\hline
Randh�ufigkeit&507&279&214&$\mathbf{1\:000}$\\\hline
\end{tabular}
\end{center}
\end{table}
%
\subsection*{Dateien}
\subsubsection*{berufsport.csv}
Die Datei enth�lt die Kontingenztabelle wie in Tabelle \ref{berufsport:urliste}.
%
\subsubsection*{berufsport.json}
Die Werte f�r beide Merkmale der Kontingenztabelle sind in der Liste \lstinline$Daten$
abgelegt. Dabei wurde die Matrix zeilenweise abgespeichert. Die Auspr�gungen
des Merkmals "`Berufsgruppe"' sind in der Liste \lstinline$KategorienX$ abgelegt, in der
Reihenfolge wie in der ersten Spalte der Kontingenztabelle. Die Auspr�gungen
des Merkmals "`Sportliche Bet�tigung"' befinden sich in der Liste \lstinline$KategorienY$.
Die Reihenfolge h�lt sich an die Angaben in der zweiten Zeile der Kontingenztabelle.

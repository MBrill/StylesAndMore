% --------------------------------------------------------------------------------
%    Farben, Optionen, Packages und Kommandos f�r Folien
% --------------------------------------------------------------------------------
\typeout{Beamder-Folien an der Hochschule Kaiserslautern}
\typeout{     (C) Manfred Brill}
\typeout{     Version 1.1 September 2018}
\typeout{     PDF-Support f�r LaTeX-Grafiken}
\typeout{     Bei Verwendung von dvi -> ps -> pdf mb.sty verwenden!}
\usepackage[german]{babel}
\usepackage[ansinew]{inputenc}
\usepackage[T1]{fontenc}
\usepackage{times}
\usepackage{amsmath}
\usepackage{amsfonts}
\usepackage{amssymb}
\usepackage{amscd}
\usepackage{mbmath}
\usepackage{color}
\usepackage{graphicx}
\usepackage{xpicture}
\usepackage{float}
% Paket f�r Hervorhebungen, Durchstreichen, ...
\usepackage[normalem]{ulem}
% Coole Symbole, wie smileys, B�gelzeichen, ...
\usepackage{marvosym}
% Paket f�r Euro-Symbol. Danach gibt es die Funktion \euro{} f�r das
% Symbol; und \EUR{1,10} f�r einen Betrag.
% F�r den Text 1,10 � verwendet man \EUR{1,10}. Die Option left f�hrt
% dazu, dass der Text als � 1,10 ausgegeben wird.
\usepackage[right]{eurosym}
% ifthen f�r Ein- und Ausblenden der L�sungen.
\usepackage{ifthen}
% Mehr Kontrollen �ber Tabellen
\usepackage{array}
% Paket f�r Filme
\usepackage{multimedia}
\usepackage{pgfpages}
% Keine subsections im Inhaltsverzeichnis
\setcounter{tocdepth}{1}
% --- Dateien aus texmf-local/tex/latex/MB
% # colors
% # variablen
% # coordinateSystems
% # beamer
% Farben
% ---------------------------------------------------------------------------------------
% Farben, f�r Folien und andere Dinge.
% Version vom Juli 2015 mit neuer Fachbereichsfarbe
%
% Letzte �nderung: 1.9.2015
%
% Um diese Datei zu verwenden muss Sie in texmf-local/MB kopiert werden
% und TeX muss aktualisiert werden!
% ---------------------------------------------------------------------------------------
\definecolor{light}{gray}{.75}
\definecolor{LightGray}{rgb}{0.24,0.24,0.24}
%   Farben rot gr�n blau f�r Koordinatenachsen
\definecolor{wred}{rgb}{0.8, 0.0, 0.0}
\definecolor{wgreen}{rgb}{0.0, 0.8, 0.0}
\definecolor{wblue}{rgb}{0.0, 0.0, 0.8}
\definecolor{wyellow}{rgb}{1.0, 1.0, 0.0}
\definecolor{wcyan}{rgb}{0.0, 1.0, 1.0}
% Fachbereichsfarbe neu
\definecolor{colorDepartment}{rgb}{0.16, 0.71, 0.86}
% Listings
\definecolor{lstback}{gray}{0.85}
%

% Variablen f�r Namen, Software, ...
% -------------------------------------------------------------------
% variablen.tex
% Variablen f�r Werkzeuge, Begriffe, Firmen
% Liegt zentral in texmf-local/tex/latex/MB
% -------------------------------------------------------------------
% Programmiersprachen
\newcommand{\java}{\texttt{Java}}
\newcommand{\cpp}{\texttt{C++}}
\newcommand{\cl}{\texttt{C}}
\newcommand{\csharp}{\texttt{C\#}}
\newcommand{\php}{\texttt{PHP}}
\newcommand{\py}{\texttt{Python}}
\newcommand{\scp}{\texttt{SciPy}}
\newcommand{\nump}{\texttt{NumPy}}
\newcommand{\mpl}{\texttt{Matplotlib}}
\newcommand{\symp}{\texttt{SymPy}}
\newcommand{\jup}{\texttt{Jupyter}}
\newcommand{\perl}{\texttt{Perl}}
\newcommand{\fc}{\texttt{Fortran}}
\newcommand{\obc}{\texttt{Objective C}}
\newcommand{\js}{\texttt{JavaScript}}
\newcommand{\omp}{\texttt{OpenMP}}
\newcommand{\mpi}{\texttt{MPI}}
\newcommand{\ocl}{\texttt{OpenCL}}
\newcommand{\cuda}{\texttt{CUDA}}
% Datei-Formate
\newcommand{\html}{\texttt{HTML}}
\newcommand{\xml}{\texttt{XML}}
\newcommand{\json}{\texttt{JSON}}
\newcommand{\rtf}{\texttt{RTF}}
\newcommand{\pdf}{\texttt{PDF}}
% Versionierung
\newcommand{\sccs}{\texttt{SCCS}}
\newcommand{\rcs}{\texttt{RCS}}
\newcommand{\cvs}{\texttt{CVS}}
\newcommand{\svn}{\texttt{SVN}}
\newcommand{\subversion}{\texttt{Subversion}}
\newcommand{\git}{\texttt{Git}}
\newcommand{\github}{\texttt{GitHub}}
\newcommand{\githubDesktop}{\texttt{GitHub Desktop}}
\newcommand{\gitkraken}{\texttt{GitKraken}}
\newcommand{\gittortoise}{\texttt{TortoiseGit}}
\newcommand{\gittree}{\texttt{Sourcetree}}
\newcommand{\bitbucket}{\texttt{BitBucket}}
\newcommand{\mercurial}{\texttt{Mercurial}}
\newcommand{\pf}{\texttt{Perforce}}
\newcommand{\pfv}{\texttt{P4V}}
\newcommand{\pfc}{\texttt{p4}}
\newcommand{\pfmerge}{\texttt{P4Merge}}
\newcommand{\pfsand}{\texttt{P4.Sandbox}}
% ASCII Editoren
\newcommand{\vi}{\texttt{vi}}
\newcommand{\emacs}{\texttt{Emacs}}
% Dokumentationswerkzeuge
\newcommand{\javadoc}{\texttt{Javadoc}}
\newcommand{\xmldoc}{\texttt{XMLdoc}}
\newcommand{\sandcastle}{\texttt{Sandcastle}}
\newcommand{\doxy}{\texttt{Doxygen}}
\newcommand{\doxyVersion}{\texttt{$1.9.2}}
\newcommand{\doxywizard}{\texttt{Doxywizard}}
\newcommand{\graphviz}{\texttt{Graphviz}}
% Logging
\newcommand{\logj}{\texttt{log4j}}
\newcommand{\lognet}{\texttt{log4net}}
\newcommand{\slfj}{\texttt{SLF4J}}
\newcommand{\logback}{\texttt{LOGBack}}
\newcommand{\jul}{\texttt{Java Logging API}}
\newcommand{\commonsLog}{\texttt{Apache Commons Logging}}
\newcommand{\boostlog}{\texttt{Boost.Log}}
\newcommand{\glog}{\texttt{Google Logging Library}}
% Testen
\newcommand{\junit}{\texttt{JUnit}}
\newcommand{\xunit}{\texttt{xUnit}}
\newcommand{\nunit}{\texttt{NUnit}}
\newcommand{\xunitnet}{\texttt{xUnit.net}}
\newcommand{\ctest}{\texttt{CTest}}
% Build
\newcommand{\ssh}{\texttt{ssh}}
\newcommand{\sftp}{\texttt{sftp}}
\newcommand{\make}{\texttt{make}}
\newcommand{\ant}{\texttt{Ant}}
\newcommand{\maven}{\texttt{Maven}}
\newcommand{\gradle}{\texttt{Gradle}}
\newcommand{\msbuild}{\texttt{MSBuild}}
\newcommand{\mstest}{\texttt{MSTest}}
\newcommand{\cmake}{\texttt{CMake}}
\newcommand{\jenkins}{\texttt{Jenkins}}
\newcommand{\actions}{\texttt{Actions}}
\newcommand{\ccontrol}{\texttt{cruisecontrol}}
\newcommand{\cdash}{\texttt{CDash}}
% IDEs
\newcommand{\eclipse}{\texttt{Eclipse}}
\newcommand{\vs}{\texttt{Visual Studio}}
\newcommand{\as}{\texttt{Android Studio}}
\newcommand{\mono}{\texttt{Mono}}
\newcommand{\monodev}{\texttt{MonoDevelop}}
% etc
\newcommand{\openO}{\texttt{OpenOffice}}
\newcommand{\Qt}{\texttt{Qt}}
\newcommand{\firefox}{\texttt{Mozilla Firefox}}
\newcommand{\kde}{\texttt{KDE}}
% Firmen und Organisationen
\newcommand{\khronos}{\texttt{Khronos}}
\newcommand{\ms}{\texttt{Microsoft}}
\newcommand{\google}{\texttt{Google}}
\newcommand{\autodesk}{\texttt{Autodesk}}
\newcommand{\sun}{\texttt{SUN}}
\newcommand{\ora}{\texttt{Oracle}}
\newcommand{\apache}{\texttt{Apache}}
% Betriebssysteme
\newcommand{\windows}{\texttt{Windows}}
\newcommand{\cyg}{\texttt{Cygwin}}
\newcommand{\linux}{\texttt{Linux}}
\newcommand{\unix}{\texttt{Unix}}
\newcommand{\osx}{\texttt{MacOS X}}
\newcommand{\android}{\texttt{Android}}
\newcommand{\net}{\texttt{.NET}}
% VR Software
\newcommand{\unity}{\texttt{Unity}}
\newcommand{\uxr}{\texttt{Unity XR}}
\newcommand{\unityHub}{\texttt{Unity Hub}}
\newcommand{\verUnity}{\texttt{2020.3.0f1  LTS}}
\newcommand{\verUHub}{\texttt{$2.4.2$}}
\newcommand{\steam}{\texttt{STEAM}}
\newcommand{\valve}{\texttt{Valve}}
\newcommand{\oculus}{\texttt{Oculus}}
\newcommand{\rift}{\texttt{Rift}}
\newcommand{\htc}{\texttt{HTC}}
\newcommand{\vive}{\texttt{Vive}}
\newcommand{\vivepro}{\texttt{Vive Pro}}
\newcommand{\fcplus}{\texttt{Focus Plus}}
\newcommand{\unreal}{\texttt{Unreal}}
\newcommand{\verUnreal}{\texttt{4}}
\newcommand{\godot}{\texttt{Godot}}
\newcommand{\verGodot}{\texttt{3.2}}
\newcommand{\cardboard}{\texttt{Cardboard}}
\newcommand{\hololens}{\texttt{Hololens 2}}
\newcommand{\juggler}{\texttt{VRJuggler}}
\newcommand{\cavelib}{\texttt{CAVELib}}
\newcommand{\gadgeteer}{\texttt{Gadgeteer}}
\newcommand{\vrpn}{\texttt{VRPN}}
\newcommand{\mvr}{\texttt{MiddleVR}}
\newcommand{\verMvr}{\texttt{$2.2$}}
\newcommand{\viveInput}{\texttt{VIVE Input Utility}} % legacy
\newcommand{\viu}{\texttt{VIVE Input Utility}}
\newcommand{\openxr}{\texttt{OpenXR}}
\newcommand{\webxr}{\texttt{WebXR}}
\newcommand{\openvr}{\texttt{OpenVR}}
\newcommand{\webvr}{\texttt{WebVR}}
\newcommand{\msxr}{\texttt{Mixed Reality}}
\newcommand{\schneider}{\texttt{Schneider}}
\newcommand{\plura}{\texttt{PluraView}}
% Computergrafik und Werkzeuge
\newcommand{\direct}{\texttt{Direct3D}}
\newcommand{\gl}{\texttt{OpenGL}}
\newcommand{\para}{\texttt{ParaView}}
\newcommand{\vtk}{\texttt{VTK}}
\newcommand{\maya}{\texttt{Maya}}
\newcommand{\dsm}{\texttt{3DS Max}}

% R und anderes zur Datenanalyse
% \R ist schon definiert (Mathematik!)
\newcommand{\Rsoft}{\texttt{R}}
\newcommand{\Rstudio}{\texttt{RStudio}}
\newcommand{\verR}{\texttt{$4.0.3$}}
\newcommand{\verRstudio}{\texttt{1.31093}}
\newcommand{\rmark}{\texttt{R Markdown}}
\newcommand{\ggplot}{\texttt{ggplot2}}
\newcommand{\plotly}{\texttt{plotly}}
\newcommand{\tidyverse}{\texttt{Tidyverse}}
\newcommand{\htmlwidgets}{\texttt{htmlwidgets}}
\newcommand{\shiny}{\texttt{Shiny}}
\newcommand{\mondrian}{\texttt{Mondrian}}
\newcommand{\tableau}{\texttt{Tableau}}
% LaTeX
\newcommand{\jabref}{\texttt{JabRef}}
\newcommand{\bibtex}{\textsc{Bib}\negthinspace\TeX}

% Saveboxes f�r das Zeichnen von Koordinatensystemen
% -----------------------------------------------------------------------------
%  Save-Boxes f�r Folien, in denen Koordinatensysteme und Karo-Muster
%  verwendet werden sollen.
%
%  Autor:   MB
%  History: 19.02.2009 Erstellt aus leeren Folien "Transformationen"
%           06.11.2012 Polarplots hinzugef�gt
%           26.01.2016 in texmf-local kopiert.
% -----------------------------------------------------------------------------
\newsavebox{\gridLines}
\savebox{\gridLines}{%
\begin{picture}(8.5,5.5)(-4.25,-2.2)
%\thicklines
\color[gray]{0.8}
\multiput(-4, -2)(0, 0.5){6}{\line(1,0){8.0}}
\multiput(-4, 0.5)(0, 0.5){6}{\line(1,0){8.0}}
\multiput(-4, -2)(0.5, 0){17}{\line(0,1){5.0}}
\color[gray]{0}
\end{picture}}% Ende \gridLines
%%%%%%%%%%%%%%%%%%%%%%%%%%%%%%%%%%%%%%%%%%%%%%%%%%%%%%%%%%%%%%%%%%%
\newsavebox{\centeredCS}
\savebox{\centeredCS}{%
\begin{picture}(8.5,5.5)(-4.25,-2.2)
%\thicklines
\color[gray]{0.8}
\multiput(-4, -2)(0, 0.5){6}{\line(1,0){8.0}}
\multiput(-4, 0.5)(0, 0.5){6}{\line(1,0){8.0}}
\multiput(-4, -2)(0.5, 0){8}{\line(0,1){5.0}}
\multiput(0.5, -2)(0.5, 0){8}{\line(0,1){5.0}}
\color[gray]{0}
\put(0,-2.2){\vector(0,1){5.5}}
\put(-4.25,0){\vector(1,0){8.5}}
\end{picture}}% Ende \centeredCS
%%%%%%%%%%%%%%%%%%%%%%%%%%%%%%%%%%%%%%%%%%%%%%%%%%%%%%%%%%%%%%%%%%%
\newsavebox{\positiveYCS}
\savebox{\positiveYCS}{%
\begin{picture}(8.5,4.5)(-4.25,-0.2)
%\thicklines
\color[gray]{0.8}
\multiput(-4, 0.5)(0, 0.5){8}{\line(1,0){8.0}}
\multiput(-4, 0.0)(0.5, 0){8}{\line(0,1){4.0}}
\multiput(0.5, 0.0)(0.5, 0){8}{\line(0,1){4.0}}
\color[gray]{0}
\put(0,-0.2){\vector(0,1){4.5}}
\put(-4.25,0){\vector(1,0){8.5}}
\end{picture}}% Ende \positiveYCS
%%%%%%%%%%%%%%%%%%%%%%%%%%%%%%%%%%%%%%%%%%%%%%%%%%%%%%%%%%%%%%%%%%%
\newsavebox{\firstquadrantCSsmall}
\savebox{\firstquadrantCSsmall}{%
\begin{picture}(8.6,5.5)(-0.2,-0.2)
\color[gray]{0.8}
\multiput(0, 0.25)(0, 0.25){20}{\line(1,0){8.0}}
\multiput(0.25, 0)(0.25, 0){32}{\line(0,1){5.0}}
\color[gray]{0}
\put(0,-0.2){\vector(0,1){5.5}}
\put(-0.2,0){\vector(1,0){8.6}}
\end{picture}}% Ende \firstquadrantCSsmall
%%%%%%%%%%%%%%%%%%%%%%%%%%%%%%%%%%%%%%%%%%%%%%%%%%%%%%%%%%%%%%%%%%%
\newsavebox{\firstquadrantCS}
\savebox{\firstquadrantCS}{%
\begin{picture}(8.6,5.5)(-0.2,-0.2)
\color[gray]{0.8}
\multiput(0, 0.5)(0, 0.5){10}{\line(1,0){8.0}}
\multiput(0.5, 0)(0.5, 0){16}{\line(0,1){5.0}}
\color[gray]{0}
\put(0,-0.2){\vector(0,1){5.5}}
\put(-0.2,0){\vector(1,0){8.6}}
\end{picture}}% Ende \firstquadrantCS
%%%%%%%%%%%%%%%%%%%%%%%%%%%%%%%%%%%%%%%%%%%%%%%%%%%%%%%%%%%%%%%%%%%
\newsavebox{\axis}
\savebox{\axis}{%
\begin{picture}(8.5,5.5)(-4.2,-2.75)
%\thicklines
\color[gray]{0}
\put(0,-2.75){\vector(0,1){5.5}}
\put(-4.25,0){\vector(1,0){8.5}}
\end{picture}}% Ende \axis
%%%%%%%%%%%%%%%%%%%%%%%%%%%%%%%%%%%%%%%%%%%%%%%%%%%%%%%%%%%%%%%%%%%
\newsavebox{\axisWithNumbers}
\savebox{\axisWithNumbers}{%
\begin{picture}(8.5,5.5)(-4.2,-2.75)
%\thicklines
\color[gray]{0}
\put(0,-2.75){\vector(0,1){5.5}}
\put(-4.25,0){\vector(1,0){8.5}}
% Beschriftung
\multiput(1, -0.1)(1,0){3}{\line(0,1){0.2}}
\multiput(-3, -0.1)(1,0){3}{\line(0,1){0.2}}
\multiput(-0.1, 1)(0,1){2}{\line(1,0){0.2}}
\multiput(-0.1, -2)(0,1){2}{\line(1,0){0.2}}
%
\small
\put(0.95, -0.4){$1$}
\put(-0.4, 0.95){$1$}
\end{picture}}% Ende \axisWithNumbers
%%%%%%%%%%%%%%%%%%%%%%%%%%%%%%%%%%%%%%%%%%%%%%%%%%%%%%%%%%%%%%%%%%%
\newsavebox{\shortAxis}
\savebox{\shortAxis}{%
\begin{picture}(3.4,3.4)(-1.7,-1.7)
%\thicklines
\color[gray]{0}
\put(0,-1.7){\vector(0,1){3.4}}
\put(-1.7,0){\vector(1,0){3.4}}
\end{picture}}% Ende \shortAxis
%%%%%%%%%%%%%%%%%%%%%%%%%%%%%%%%%%%%%%%%%%%%%%%%%%%%%%%%%%%%%%%%%%%
\newsavebox{\shortAxisWithNumbers}
\savebox{\shortAxisWithNumbers}{%
\begin{picture}(3.4,3.4)(-1.7,-1.7)
%\thicklines
\color[gray]{0}
\put(0,-1.7){\vector(0,1){3.4}}
\put(-1.7,0){\vector(1,0){3.4}}
% Beschriftung
\put(1, -0.1){\line(0,1){0.2}}
\put(-1, -0.1){\line(0,1){0.2}}
\put(-0.1, 1){\line(1,0){0.2}}
\put(-0.1, -1){\line(1,0){0.2}}
%
\small
\put(0.95, -0.4){$1$}
\put(-0.4, 0.95){$1$}
\end{picture}}% Ende \shortAxisWithNumbers
%%%%%%%%%%%%%%%%%%%%%%%%%%%%%%%%%%%%%%%%%%%%%%%%%%%%%%%%%%%%%%%%%%%
\newsavebox{\positiveAxis}
\savebox{\positiveAxis}{%
\begin{picture}(5.5,5.5)(-0.2,-0.2)
\put(0,-0.2){\vector(0,1){5.5}}
\put(-0.2,0){\vector(1,0){5.5}}
\end{picture}}% Ende \positiveAxis
%%%%%%%%%%%%%%%%%%%%%%%%%%%%%%%%%%%%%%%%%%%%%%%%%%%%%%%%%%%%%%%%%%%
\newsavebox{\shortPositiveAxis}
\savebox{\shortPositiveAxis}{%
\begin{picture}(3.5,3.5)(-0.2,-0.2)
\put(0,-0.2){\vector(0,1){3.5}}
\put(-0.2,0){\vector(1,0){3.5}}
\end{picture}}% Ende \shortPositiveAxis
%%%%%%%%%%%%%%%%%%%%%%%%%%%%%%%%%%%%%%%%%%%%%%%%%%%%%%%%%%%%%%%%%%%
\newsavebox{\positiveAxisWithNumbers}
\savebox{\positiveAxisWithNumbers}{%
\begin{picture}(5.5,5.5)(-0.2,-0.2)
\put(0,-0.2){\vector(0,1){5.5}}
\put(-0.2,0){\vector(1,0){5.5}}
% Beschriftung
\multiput(1, -0.1)(1,0){4}{\line(0,1){0.2}}
\multiput(-0.1, 1)(0,1){4}{\line(1,0){0.2}}
%
\small
\put(0.95, -0.4){$1$}
\put(-0.4, 0.95){$1$}
\end{picture}}% Ende \positiveAxisWithNumbers
%%%%%%%%%%%%%%%%%%%%%%%%%%%%%%%%%%%%%%%%%%%%%%%%%%%%%%%%%%%%%%%%%%%
\newsavebox{\polarCircles}
\savebox{\polarCircles}{%
\begin{picture}(6.5,6.5)(-3.25,-3.25)
%\thicklines
\color[gray]{0.8}
% Argument bei circle ist der Durchmesser!
\put(0,0){\circle{2.0}}
\put(0,0){\circle{4.0}}
\put(0,0){\circle{6.0}}
\color[gray]{0}
\end{picture}}% Ende \polarCircles
%%%%%%%%%%%%%%%%%%%%%%%%%%%%%%%%%%%%%%%%%%%%%%%%%%%%%%%%%%%%%%%%%%%
\newsavebox{\polarCirclesFine}
\savebox{\polarCirclesFine}{%
\begin{picture}(6.5,6.5)(-3.25,-3.25)
%\thicklines
\color[gray]{0.8}
% Argument bei circle ist der Durchmesser!
\put(0,0){\circle{1.0}}
\put(0,0){\circle{2.0}}
\put(0,0){\circle{3.0}}
\put(0,0){\circle{4.0}}
\put(0,0){\circle{5.0}}
\put(0,0){\circle{6.0}}
\put(0,0){\circle{7.0}}
\color[gray]{0}
\end{picture}}% Ende \polarCirclesFine
%%%%%%%%%%%%%%%%%%%%%%%%%%%%%%%%%%%%%%%%%%%%%%%%%%%%%%%%%%%%%%%%%%%
\newsavebox{\polarAxis}
\savebox{\polarAxis}{%
\begin{picture}(6.5,6.5)(-3.25,-3.25)
\put(-2.12,-2.12){\line(1,1){4.24}}
\put(2.12, -2.12){\line(-1, 1){4.24}}
\thicklines % damit die "normalen" Striche auch gleich dick sind ...
\put(-3,0){\line(1,0){6.0}}
\put(0,-3){\line(0,1){6.0}}
\end{picture}}% Ende \polarAxis
%%%%%%%%%%%%%%%%%%%%%%%%%%%%%%%%%%%%%%%%%%%%%%%%%%%%%%%%%%%%%%%%%%%
\newsavebox{\polarAxisFine}
\savebox{\polarAxisFine}{%
\begin{picture}(6.5,6.5)(-3.25,-3.25)
\put(-2.47,-2.47){\line(1,1){4.94}}
\put(2.47, -2.47){\line(-1, 1){4.94}}
\thicklines % damit die "normalen" Striche auch gleich dick sind ...
\put(-3.5,0){\line(1,0){7.0}}
\put(0,-3.5){\line(0,1){7.0}}
\end{picture}}% Ende \polarAxisFine
%%%%%%%%%%%%%%%%%%%%%%%%%%%%%%%%%%%%%%%%%%%%%%%%%%%%%%%%%%%%%%%%%%%
\newsavebox{\polarPlot}
\savebox{\polarPlot}{%
\begin{picture}(6.5,6.5)(-3.25,-3.25)
\put(-3.25,-3.25){\usebox{\polarAxis}}
\put(-3.25,-3.25){\usebox{\polarCircles}}
\end{picture}}% Ende \polarPlot
%%%%%%%%%%%%%%%%%%%%%%%%%%%%%%%%%%%%%%%%%%%%%%%%%%%%%%%%%%%%%%%%%%% \newsavebox{\polarPlot}
\newsavebox{\polarPlotFine}
\savebox{\polarPlotFine}{%
\begin{picture}(6.5,6.5)(-3.25,-3.25)
\put(-3.25,-3.25){\usebox{\polarAxisFine}}
\put(-3.25,-3.25){\usebox{\polarCirclesFine}}
\end{picture}}% Ende \polarPlotFine
%%%%%%%%%%%%%%%%%%%%%%%%%%%%%%%%%%%%%%%%%%%%%%%%%%%%%%%%%%%%%%%%%%%

% Beamer theme
\usetheme{CambridgeUS}
\chapter{Folien mit dem Beamer-Package}
Folien werden mit Hilfe des \texttt{Beamer}-Package
erstellt. Dabei stehen die Quellen der Folien parallel zu den Texten und �bungsaufgaben.
Damit wird gew�hrleistet, dass Abbildungen und andere Assets m�glichst  wiederverwendet werden.

Wie schon f�r die Texte gibt es auch hier eine Datei  mit
Angaben, die abh�ngig von der jeweiligen Vorlesung sind. Diese Datei hei�t \datei{header.tex}.
Das folgende Beispiel
stammt aus den Folien f�r die Pr�senzphase des Fachs \emph{Stochastik f�r Informatiker}
im Studiengang IT-Analyst:
\begin{lstlisting}{}
\newcommand{\lectureName}{Stochastik f�r Informatiker}
% Titel der Folie
\title{\lectureName}
\author{Prof. Dr. Manfred Brill}
\institute{Hochschule Kaiserslautern\\
     Fachbereich Informatik und Mikrosystemtechnik}
\date[]{}
% Variablen wie Semester ...
%
\newcommand{\theSemester}{Sommersemester~2018}
% Standardverzeichnis f�r das Basis-Verzeichnis der Bilder
%
\newcommand{\imagePath}{../../images}
%
% Name der Bild-Datei, die auf die Frage-Folie kommen soll
\newcommand{\questionImage}{Misc/Answer_to_Life}
\end{lstlisting}
Die Bezeichnung der Lehrveranstaltung wird definiert und anschlie�end werden die Variablen
f�r die Titelfolie gesetzt.
Anschlie�end definieren wir das aktuelle Semester und die Pfade f�r das Verzeichnis, in dem Bilder f�r die Lehrveranstaltung liegen. Jede Folie enth�lt eine letzte Folie mit einem Bild. Das Bild, das verwendet werden soll wird in der Variable \lstinline$questionImage$
angegeben.

An die Stelle der Datei \texttt{setup.tex} tritt
\datei{slidesheader.tex}, die im Verzeichnis \datei{texmf-local} steht und
unabh�ngig von den Lehrveranstaltungen verwendet werden kann.
Diese Datei wird im Folgenden noch beschrieben.

Bevor wir auf \datei{slidesheader.tex} eingehen werfen wir einen Blick
auf \datei{keywords}. Hier werden Schlagw�rter und andere Informationen f�r
die Eigenschaften des erstellten \pdf{}-Dokuments gespeichert. Das Vorgehen ist
�hnlich wie bei den Texten. Hier ein Beispiel,
wieder aus der gleichen Veranstaltung:

\begin{lstlisting}{}
\subject{Stochastik f�r Informatiker}
\keywords{Wahrscheinlichkeitsrechnung und Statistik,
          Medieninformatik, Hochschule Kaiserslautern}
%
\usepackage{breakurl}
\hypersetup{
pdfauthor = {Manfred Brill},
pdfsubject = {Stochastik f�r Informatiker},
pdftitle = {Stochastik f�r Informatiker},
pdfkeywords = {Stochastik f�r Informatiker, IT-Analyst,
      Hochschule Kaiserslautern},
pdfpagelayout = SinglePage,
pdfpagemode = UseThumbs,
pdfdisplaydoctitle = true
}

\urlstyle{same}
\end{lstlisting}
Die Datei \datei{slidesheader.tex} l�dt �hnlich wie \datei{setup.tex}
einige Packages. Da es bei Folien durchaus vorkommen kann, dass Medien
eingebunden werden wird das Package \lstinline$multimedia$ geladen.
Neben den Folien gibt es Handouts, deshalb wird das Package
\lstinline$pgfpages$ verwendet.
Um Smileys und andere Abbildungen in Folien zu verwenden wird das Package
\lstinline$marvosym$ geladen. Auch f�r die Folien
wird in der aktuellen Version das Paket
\lstinline$xpicture$ geladen. Damit k�nnen alle Grafiken, die
bisher mit \lstinline$eepic$ und \lstinline$dvips$ erstellt wurden auch
mit PDF\LaTeX{} verwendet werden.

Wie f�r die anderen Dokumente werden die Einstellungen
in den Dateien \lstinline$colors.tex$, \lstinline$variablen.tex$ und \lstinline$coordinateSystems.tex$
geladen. Auch das \lstinline$listings$-Paket wird geladen und konfiguriert.
F�r Folien wird die Variable \lstinline$solutions$ definiert, die anschlie�end
in den Folien und auch in \lstinline$slidesheader$ verwendet wird.

Das \lstinline$beamer$-Package bietet verschiedene Styles an, die angepasst werden k�nnen.
Als \emph{theme} wird \lstinline$CambridgeUS$ ausgew�hlt. Die weiteren Anpassungen
finden in der Datei \datei{beamer.tex} statt.
Wichtig ist die Verwendung der Farbe \lstinline$colorDepartment$, die in \datei{colors.tex}
definiert ist. Hier wird die Farbe des Fachbereichs aus dem Logo der Hochschule verwendet. Das Ergebnis
ist in Abbildung \ref{titelfolie} zu sehen.

Folien die nur aus Abbildung und einer �berschrift bestehen kommen h�ufig in Pr�sentationen vor.
Deshalb ist eine Funktion
definiert, mit deren Hilfe man eine solche Folie erstellen kann:
\begin{lstlisting}{}
\newcommand{\imageslide}[3]{%
\begin{frame}{#1}%
  \begin{center}%
   \includegraphics[width = #2]{\imagePath/#3}%
  \end{center}%
\end{frame}}
\end{lstlisting}
Diese Funktion hat drei Argumente. Das erste Argument ist der Titel der Folie, gefolgt
von der Breite, mit der die Abbildung in die Folie eingef�hrt werden soll. Das letzte Argument
ist der Pfad zur Bild-Datei, relativ zum Wert der Variable \lstinline$imagePath$.

\begin{lstlisting}{}
\imageslide{Die Startseite}{6cm}{OLAT/itastochastik_index}
\end{lstlisting}
In Abbildung \ref{bildfolie} ist das Ergebnis dieser Anweisung zu sehen.

\begin{figure}[ht]
\begin{center}
\includegraphics[width=8cm]{\imagePath/imageFolienBeispiel}
\caption{\label{bildfolie}Eine Folie, erzeugt mit \lstinline$imageslide$}
\end{center}
\end{figure}
Am Ende jeder Datei mit Folien wird eine Folie eingef�gt, die die Zuh�rer zu Fragen auffordert. Die H�rer
erkennen so immer, dass das Thema abgeschlossen ist. Daf�r gibt es die Funktion \lstinline$questions$.
Als Bild wird die Datei verwendet, die in \datei{header.tex} in der Variable \lstinline$\questionImage$
definiert ist. Die Funktion \lstinline$questions$ reagiert auf die Variable \lstinline$solutions$.
Die Folie wird nur erzeugt, wenn \lstinline$solutions$ auf \lstinline$TRUE$ gesetzt ist. So kann
die Fragefolie bei Handouts und anderen Dokumenten ausgeblendet werden. F�r die Studierenden
macht es sicher keinen Sinn, in $50$ PDF-Dateien jedes Mal die Fragefolie zu drucken.
Abbildung \ref{fragefolie} zeigt ein Ergebnis der Funktion.

\begin{figure}[ht]
\begin{center}
\includegraphics[width=8cm]{\imagePath/questionsSlide}
\caption{\label{fragefolie}Eine Folie, erzeugt mit \lstinline$questions$}
\end{center}
\end{figure}
Die Breite des eingef�gten Bilds kann als optionales Argument �bergeben werden; der Default-Wert daf�r ist
$5$~cm. Ein Bild mit einer Breite von $8$~cm wird so eingef�gt:
\begin{lstlisting}{}
\questions[8cm]
\end{lstlisting}

Mit den beschriebenen Konfigurationen
sieht der Kopf der \TeX{}-Datei f�r die Folien immer gleich aus:
\begin{lstlisting}{}
\documentclass{beamer}
% --------------------------------------------------------------------------------
%   Variable f�r Folien
% --------------------------------------------------------------------------------
% Funktion f�r Footer/Header in den Folien
\newcommand{\lectureName}{Dokumentation coordinateSystems.tex}
% Titel der Folie
\title{\lectureName}
\author{Manfred Brill}
\institute{Hochschule Kaiserslautern\\Fachbereich Informatik und Mikrosystemtechnik}
\date[]{}
% Semester
\newcommand{\theSemester}{Sommersemester~2018}
% Standardverzeichnis f�r das Basisverzeichnis der Bilder
%
\newcommand{\imagePath}{../../generalImages}
%
% Name der Bilddatei, die auf die Fragefolie kommen soll
\newcommand{\questionImage}{misc/ada} 
% --------------------------------------------------------------------------------
%    Farben, Optionen, Packages und Kommandos f�r Folien
% --------------------------------------------------------------------------------
\typeout{Beamder-Folien an der Hochschule Kaiserslautern}
\typeout{     (C) Manfred Brill}
\typeout{     Version 1.1 September 2018}
\typeout{     PDF-Support f�r LaTeX-Grafiken}
\typeout{     Bei Verwendung von dvi -> ps -> pdf mb.sty verwenden!}
\usepackage[german]{babel}
\usepackage[ansinew]{inputenc}
\usepackage[T1]{fontenc}
\usepackage{times}
\usepackage{amsmath}
\usepackage{amsfonts}
\usepackage{amssymb}
\usepackage{amscd}
\usepackage{mbmath}
\usepackage{color}
\usepackage{graphicx}
\usepackage{xpicture}
\usepackage{float}
% Paket f�r Hervorhebungen, Durchstreichen, ...
\usepackage[normalem]{ulem}
% Coole Symbole, wie smileys, B�gelzeichen, ...
\usepackage{marvosym}
% Paket f�r Euro-Symbol. Danach gibt es die Funktion \euro{} f�r das
% Symbol; und \EUR{1,10} f�r einen Betrag.
% F�r den Text 1,10 � verwendet man \EUR{1,10}. Die Option left f�hrt
% dazu, dass der Text als � 1,10 ausgegeben wird.
\usepackage[right]{eurosym}
% ifthen f�r Ein- und Ausblenden der L�sungen.
\usepackage{ifthen}
% Mehr Kontrollen �ber Tabellen
\usepackage{array}
% Paket f�r Filme
\usepackage{multimedia}
\usepackage{pgfpages}
% Keine subsections im Inhaltsverzeichnis
\setcounter{tocdepth}{1}
% --- Dateien aus texmf-local/tex/latex/MB
% # colors
% # variablen
% # coordinateSystems
% # beamer
% Farben
% ---------------------------------------------------------------------------------------
% Farben, f�r Folien und andere Dinge.
% Version vom Juli 2015 mit neuer Fachbereichsfarbe
%
% Letzte �nderung: 1.9.2015
%
% Um diese Datei zu verwenden muss Sie in texmf-local/MB kopiert werden
% und TeX muss aktualisiert werden!
% ---------------------------------------------------------------------------------------
\definecolor{light}{gray}{.75}
\definecolor{LightGray}{rgb}{0.24,0.24,0.24}
%   Farben rot gr�n blau f�r Koordinatenachsen
\definecolor{wred}{rgb}{0.8, 0.0, 0.0}
\definecolor{wgreen}{rgb}{0.0, 0.8, 0.0}
\definecolor{wblue}{rgb}{0.0, 0.0, 0.8}
\definecolor{wyellow}{rgb}{1.0, 1.0, 0.0}
\definecolor{wcyan}{rgb}{0.0, 1.0, 1.0}
% Fachbereichsfarbe neu
\definecolor{colorDepartment}{rgb}{0.16, 0.71, 0.86}
% Listings
\definecolor{lstback}{gray}{0.85}
%

% Variablen f�r Namen, Software, ...
% -------------------------------------------------------------------
% variablen.tex
% Variablen f�r Werkzeuge, Begriffe, Firmen
% Liegt zentral in texmf-local/tex/latex/MB
% -------------------------------------------------------------------
% Programmiersprachen
\newcommand{\java}{\texttt{Java}}
\newcommand{\cpp}{\texttt{C++}}
\newcommand{\cl}{\texttt{C}}
\newcommand{\csharp}{\texttt{C\#}}
\newcommand{\php}{\texttt{PHP}}
\newcommand{\py}{\texttt{Python}}
\newcommand{\scp}{\texttt{SciPy}}
\newcommand{\nump}{\texttt{NumPy}}
\newcommand{\mpl}{\texttt{Matplotlib}}
\newcommand{\symp}{\texttt{SymPy}}
\newcommand{\jup}{\texttt{Jupyter}}
\newcommand{\perl}{\texttt{Perl}}
\newcommand{\fc}{\texttt{Fortran}}
\newcommand{\obc}{\texttt{Objective C}}
\newcommand{\js}{\texttt{JavaScript}}
\newcommand{\omp}{\texttt{OpenMP}}
\newcommand{\mpi}{\texttt{MPI}}
\newcommand{\ocl}{\texttt{OpenCL}}
\newcommand{\cuda}{\texttt{CUDA}}
% Datei-Formate
\newcommand{\html}{\texttt{HTML}}
\newcommand{\xml}{\texttt{XML}}
\newcommand{\json}{\texttt{JSON}}
\newcommand{\rtf}{\texttt{RTF}}
\newcommand{\pdf}{\texttt{PDF}}
% Versionierung
\newcommand{\sccs}{\texttt{SCCS}}
\newcommand{\rcs}{\texttt{RCS}}
\newcommand{\cvs}{\texttt{CVS}}
\newcommand{\svn}{\texttt{SVN}}
\newcommand{\subversion}{\texttt{Subversion}}
\newcommand{\git}{\texttt{Git}}
\newcommand{\github}{\texttt{GitHub}}
\newcommand{\githubDesktop}{\texttt{GitHub Desktop}}
\newcommand{\gitkraken}{\texttt{GitKraken}}
\newcommand{\gittortoise}{\texttt{TortoiseGit}}
\newcommand{\gittree}{\texttt{Sourcetree}}
\newcommand{\bitbucket}{\texttt{BitBucket}}
\newcommand{\mercurial}{\texttt{Mercurial}}
\newcommand{\pf}{\texttt{Perforce}}
\newcommand{\pfv}{\texttt{P4V}}
\newcommand{\pfc}{\texttt{p4}}
\newcommand{\pfmerge}{\texttt{P4Merge}}
\newcommand{\pfsand}{\texttt{P4.Sandbox}}
% ASCII Editoren
\newcommand{\vi}{\texttt{vi}}
\newcommand{\emacs}{\texttt{Emacs}}
% Dokumentationswerkzeuge
\newcommand{\javadoc}{\texttt{Javadoc}}
\newcommand{\xmldoc}{\texttt{XMLdoc}}
\newcommand{\sandcastle}{\texttt{Sandcastle}}
\newcommand{\doxy}{\texttt{Doxygen}}
\newcommand{\doxyVersion}{\texttt{$1.9.2}}
\newcommand{\doxywizard}{\texttt{Doxywizard}}
\newcommand{\graphviz}{\texttt{Graphviz}}
% Logging
\newcommand{\logj}{\texttt{log4j}}
\newcommand{\lognet}{\texttt{log4net}}
\newcommand{\slfj}{\texttt{SLF4J}}
\newcommand{\logback}{\texttt{LOGBack}}
\newcommand{\jul}{\texttt{Java Logging API}}
\newcommand{\commonsLog}{\texttt{Apache Commons Logging}}
\newcommand{\boostlog}{\texttt{Boost.Log}}
\newcommand{\glog}{\texttt{Google Logging Library}}
% Testen
\newcommand{\junit}{\texttt{JUnit}}
\newcommand{\xunit}{\texttt{xUnit}}
\newcommand{\nunit}{\texttt{NUnit}}
\newcommand{\xunitnet}{\texttt{xUnit.net}}
\newcommand{\ctest}{\texttt{CTest}}
% Build
\newcommand{\ssh}{\texttt{ssh}}
\newcommand{\sftp}{\texttt{sftp}}
\newcommand{\make}{\texttt{make}}
\newcommand{\ant}{\texttt{Ant}}
\newcommand{\maven}{\texttt{Maven}}
\newcommand{\gradle}{\texttt{Gradle}}
\newcommand{\msbuild}{\texttt{MSBuild}}
\newcommand{\mstest}{\texttt{MSTest}}
\newcommand{\cmake}{\texttt{CMake}}
\newcommand{\jenkins}{\texttt{Jenkins}}
\newcommand{\actions}{\texttt{Actions}}
\newcommand{\ccontrol}{\texttt{cruisecontrol}}
\newcommand{\cdash}{\texttt{CDash}}
% IDEs
\newcommand{\eclipse}{\texttt{Eclipse}}
\newcommand{\vs}{\texttt{Visual Studio}}
\newcommand{\as}{\texttt{Android Studio}}
\newcommand{\mono}{\texttt{Mono}}
\newcommand{\monodev}{\texttt{MonoDevelop}}
% etc
\newcommand{\openO}{\texttt{OpenOffice}}
\newcommand{\Qt}{\texttt{Qt}}
\newcommand{\firefox}{\texttt{Mozilla Firefox}}
\newcommand{\kde}{\texttt{KDE}}
% Firmen und Organisationen
\newcommand{\khronos}{\texttt{Khronos}}
\newcommand{\ms}{\texttt{Microsoft}}
\newcommand{\google}{\texttt{Google}}
\newcommand{\autodesk}{\texttt{Autodesk}}
\newcommand{\sun}{\texttt{SUN}}
\newcommand{\ora}{\texttt{Oracle}}
\newcommand{\apache}{\texttt{Apache}}
% Betriebssysteme
\newcommand{\windows}{\texttt{Windows}}
\newcommand{\cyg}{\texttt{Cygwin}}
\newcommand{\linux}{\texttt{Linux}}
\newcommand{\unix}{\texttt{Unix}}
\newcommand{\osx}{\texttt{MacOS X}}
\newcommand{\android}{\texttt{Android}}
\newcommand{\net}{\texttt{.NET}}
% VR Software
\newcommand{\unity}{\texttt{Unity}}
\newcommand{\uxr}{\texttt{Unity XR}}
\newcommand{\unityHub}{\texttt{Unity Hub}}
\newcommand{\verUnity}{\texttt{2020.3.0f1  LTS}}
\newcommand{\verUHub}{\texttt{$2.4.2$}}
\newcommand{\steam}{\texttt{STEAM}}
\newcommand{\valve}{\texttt{Valve}}
\newcommand{\oculus}{\texttt{Oculus}}
\newcommand{\rift}{\texttt{Rift}}
\newcommand{\htc}{\texttt{HTC}}
\newcommand{\vive}{\texttt{Vive}}
\newcommand{\vivepro}{\texttt{Vive Pro}}
\newcommand{\fcplus}{\texttt{Focus Plus}}
\newcommand{\unreal}{\texttt{Unreal}}
\newcommand{\verUnreal}{\texttt{4}}
\newcommand{\godot}{\texttt{Godot}}
\newcommand{\verGodot}{\texttt{3.2}}
\newcommand{\cardboard}{\texttt{Cardboard}}
\newcommand{\hololens}{\texttt{Hololens 2}}
\newcommand{\juggler}{\texttt{VRJuggler}}
\newcommand{\cavelib}{\texttt{CAVELib}}
\newcommand{\gadgeteer}{\texttt{Gadgeteer}}
\newcommand{\vrpn}{\texttt{VRPN}}
\newcommand{\mvr}{\texttt{MiddleVR}}
\newcommand{\verMvr}{\texttt{$2.2$}}
\newcommand{\viveInput}{\texttt{VIVE Input Utility}} % legacy
\newcommand{\viu}{\texttt{VIVE Input Utility}}
\newcommand{\openxr}{\texttt{OpenXR}}
\newcommand{\webxr}{\texttt{WebXR}}
\newcommand{\openvr}{\texttt{OpenVR}}
\newcommand{\webvr}{\texttt{WebVR}}
\newcommand{\msxr}{\texttt{Mixed Reality}}
\newcommand{\schneider}{\texttt{Schneider}}
\newcommand{\plura}{\texttt{PluraView}}
% Computergrafik und Werkzeuge
\newcommand{\direct}{\texttt{Direct3D}}
\newcommand{\gl}{\texttt{OpenGL}}
\newcommand{\para}{\texttt{ParaView}}
\newcommand{\vtk}{\texttt{VTK}}
\newcommand{\maya}{\texttt{Maya}}
\newcommand{\dsm}{\texttt{3DS Max}}

% R und anderes zur Datenanalyse
% \R ist schon definiert (Mathematik!)
\newcommand{\Rsoft}{\texttt{R}}
\newcommand{\Rstudio}{\texttt{RStudio}}
\newcommand{\verR}{\texttt{$4.0.3$}}
\newcommand{\verRstudio}{\texttt{1.31093}}
\newcommand{\rmark}{\texttt{R Markdown}}
\newcommand{\ggplot}{\texttt{ggplot2}}
\newcommand{\plotly}{\texttt{plotly}}
\newcommand{\tidyverse}{\texttt{Tidyverse}}
\newcommand{\htmlwidgets}{\texttt{htmlwidgets}}
\newcommand{\shiny}{\texttt{Shiny}}
\newcommand{\mondrian}{\texttt{Mondrian}}
\newcommand{\tableau}{\texttt{Tableau}}
% LaTeX
\newcommand{\jabref}{\texttt{JabRef}}
\newcommand{\bibtex}{\textsc{Bib}\negthinspace\TeX}

% Saveboxes f�r das Zeichnen von Koordinatensystemen
% -----------------------------------------------------------------------------
%  Save-Boxes f�r Folien, in denen Koordinatensysteme und Karo-Muster
%  verwendet werden sollen.
%
%  Autor:   MB
%  History: 19.02.2009 Erstellt aus leeren Folien "Transformationen"
%           06.11.2012 Polarplots hinzugef�gt
%           26.01.2016 in texmf-local kopiert.
% -----------------------------------------------------------------------------
\newsavebox{\gridLines}
\savebox{\gridLines}{%
\begin{picture}(8.5,5.5)(-4.25,-2.2)
%\thicklines
\color[gray]{0.8}
\multiput(-4, -2)(0, 0.5){6}{\line(1,0){8.0}}
\multiput(-4, 0.5)(0, 0.5){6}{\line(1,0){8.0}}
\multiput(-4, -2)(0.5, 0){17}{\line(0,1){5.0}}
\color[gray]{0}
\end{picture}}% Ende \gridLines
%%%%%%%%%%%%%%%%%%%%%%%%%%%%%%%%%%%%%%%%%%%%%%%%%%%%%%%%%%%%%%%%%%%
\newsavebox{\centeredCS}
\savebox{\centeredCS}{%
\begin{picture}(8.5,5.5)(-4.25,-2.2)
%\thicklines
\color[gray]{0.8}
\multiput(-4, -2)(0, 0.5){6}{\line(1,0){8.0}}
\multiput(-4, 0.5)(0, 0.5){6}{\line(1,0){8.0}}
\multiput(-4, -2)(0.5, 0){8}{\line(0,1){5.0}}
\multiput(0.5, -2)(0.5, 0){8}{\line(0,1){5.0}}
\color[gray]{0}
\put(0,-2.2){\vector(0,1){5.5}}
\put(-4.25,0){\vector(1,0){8.5}}
\end{picture}}% Ende \centeredCS
%%%%%%%%%%%%%%%%%%%%%%%%%%%%%%%%%%%%%%%%%%%%%%%%%%%%%%%%%%%%%%%%%%%
\newsavebox{\positiveYCS}
\savebox{\positiveYCS}{%
\begin{picture}(8.5,4.5)(-4.25,-0.2)
%\thicklines
\color[gray]{0.8}
\multiput(-4, 0.5)(0, 0.5){8}{\line(1,0){8.0}}
\multiput(-4, 0.0)(0.5, 0){8}{\line(0,1){4.0}}
\multiput(0.5, 0.0)(0.5, 0){8}{\line(0,1){4.0}}
\color[gray]{0}
\put(0,-0.2){\vector(0,1){4.5}}
\put(-4.25,0){\vector(1,0){8.5}}
\end{picture}}% Ende \positiveYCS
%%%%%%%%%%%%%%%%%%%%%%%%%%%%%%%%%%%%%%%%%%%%%%%%%%%%%%%%%%%%%%%%%%%
\newsavebox{\firstquadrantCSsmall}
\savebox{\firstquadrantCSsmall}{%
\begin{picture}(8.6,5.5)(-0.2,-0.2)
\color[gray]{0.8}
\multiput(0, 0.25)(0, 0.25){20}{\line(1,0){8.0}}
\multiput(0.25, 0)(0.25, 0){32}{\line(0,1){5.0}}
\color[gray]{0}
\put(0,-0.2){\vector(0,1){5.5}}
\put(-0.2,0){\vector(1,0){8.6}}
\end{picture}}% Ende \firstquadrantCSsmall
%%%%%%%%%%%%%%%%%%%%%%%%%%%%%%%%%%%%%%%%%%%%%%%%%%%%%%%%%%%%%%%%%%%
\newsavebox{\firstquadrantCS}
\savebox{\firstquadrantCS}{%
\begin{picture}(8.6,5.5)(-0.2,-0.2)
\color[gray]{0.8}
\multiput(0, 0.5)(0, 0.5){10}{\line(1,0){8.0}}
\multiput(0.5, 0)(0.5, 0){16}{\line(0,1){5.0}}
\color[gray]{0}
\put(0,-0.2){\vector(0,1){5.5}}
\put(-0.2,0){\vector(1,0){8.6}}
\end{picture}}% Ende \firstquadrantCS
%%%%%%%%%%%%%%%%%%%%%%%%%%%%%%%%%%%%%%%%%%%%%%%%%%%%%%%%%%%%%%%%%%%
\newsavebox{\axis}
\savebox{\axis}{%
\begin{picture}(8.5,5.5)(-4.2,-2.75)
%\thicklines
\color[gray]{0}
\put(0,-2.75){\vector(0,1){5.5}}
\put(-4.25,0){\vector(1,0){8.5}}
\end{picture}}% Ende \axis
%%%%%%%%%%%%%%%%%%%%%%%%%%%%%%%%%%%%%%%%%%%%%%%%%%%%%%%%%%%%%%%%%%%
\newsavebox{\axisWithNumbers}
\savebox{\axisWithNumbers}{%
\begin{picture}(8.5,5.5)(-4.2,-2.75)
%\thicklines
\color[gray]{0}
\put(0,-2.75){\vector(0,1){5.5}}
\put(-4.25,0){\vector(1,0){8.5}}
% Beschriftung
\multiput(1, -0.1)(1,0){3}{\line(0,1){0.2}}
\multiput(-3, -0.1)(1,0){3}{\line(0,1){0.2}}
\multiput(-0.1, 1)(0,1){2}{\line(1,0){0.2}}
\multiput(-0.1, -2)(0,1){2}{\line(1,0){0.2}}
%
\small
\put(0.95, -0.4){$1$}
\put(-0.4, 0.95){$1$}
\end{picture}}% Ende \axisWithNumbers
%%%%%%%%%%%%%%%%%%%%%%%%%%%%%%%%%%%%%%%%%%%%%%%%%%%%%%%%%%%%%%%%%%%
\newsavebox{\shortAxis}
\savebox{\shortAxis}{%
\begin{picture}(3.4,3.4)(-1.7,-1.7)
%\thicklines
\color[gray]{0}
\put(0,-1.7){\vector(0,1){3.4}}
\put(-1.7,0){\vector(1,0){3.4}}
\end{picture}}% Ende \shortAxis
%%%%%%%%%%%%%%%%%%%%%%%%%%%%%%%%%%%%%%%%%%%%%%%%%%%%%%%%%%%%%%%%%%%
\newsavebox{\shortAxisWithNumbers}
\savebox{\shortAxisWithNumbers}{%
\begin{picture}(3.4,3.4)(-1.7,-1.7)
%\thicklines
\color[gray]{0}
\put(0,-1.7){\vector(0,1){3.4}}
\put(-1.7,0){\vector(1,0){3.4}}
% Beschriftung
\put(1, -0.1){\line(0,1){0.2}}
\put(-1, -0.1){\line(0,1){0.2}}
\put(-0.1, 1){\line(1,0){0.2}}
\put(-0.1, -1){\line(1,0){0.2}}
%
\small
\put(0.95, -0.4){$1$}
\put(-0.4, 0.95){$1$}
\end{picture}}% Ende \shortAxisWithNumbers
%%%%%%%%%%%%%%%%%%%%%%%%%%%%%%%%%%%%%%%%%%%%%%%%%%%%%%%%%%%%%%%%%%%
\newsavebox{\positiveAxis}
\savebox{\positiveAxis}{%
\begin{picture}(5.5,5.5)(-0.2,-0.2)
\put(0,-0.2){\vector(0,1){5.5}}
\put(-0.2,0){\vector(1,0){5.5}}
\end{picture}}% Ende \positiveAxis
%%%%%%%%%%%%%%%%%%%%%%%%%%%%%%%%%%%%%%%%%%%%%%%%%%%%%%%%%%%%%%%%%%%
\newsavebox{\shortPositiveAxis}
\savebox{\shortPositiveAxis}{%
\begin{picture}(3.5,3.5)(-0.2,-0.2)
\put(0,-0.2){\vector(0,1){3.5}}
\put(-0.2,0){\vector(1,0){3.5}}
\end{picture}}% Ende \shortPositiveAxis
%%%%%%%%%%%%%%%%%%%%%%%%%%%%%%%%%%%%%%%%%%%%%%%%%%%%%%%%%%%%%%%%%%%
\newsavebox{\positiveAxisWithNumbers}
\savebox{\positiveAxisWithNumbers}{%
\begin{picture}(5.5,5.5)(-0.2,-0.2)
\put(0,-0.2){\vector(0,1){5.5}}
\put(-0.2,0){\vector(1,0){5.5}}
% Beschriftung
\multiput(1, -0.1)(1,0){4}{\line(0,1){0.2}}
\multiput(-0.1, 1)(0,1){4}{\line(1,0){0.2}}
%
\small
\put(0.95, -0.4){$1$}
\put(-0.4, 0.95){$1$}
\end{picture}}% Ende \positiveAxisWithNumbers
%%%%%%%%%%%%%%%%%%%%%%%%%%%%%%%%%%%%%%%%%%%%%%%%%%%%%%%%%%%%%%%%%%%
\newsavebox{\polarCircles}
\savebox{\polarCircles}{%
\begin{picture}(6.5,6.5)(-3.25,-3.25)
%\thicklines
\color[gray]{0.8}
% Argument bei circle ist der Durchmesser!
\put(0,0){\circle{2.0}}
\put(0,0){\circle{4.0}}
\put(0,0){\circle{6.0}}
\color[gray]{0}
\end{picture}}% Ende \polarCircles
%%%%%%%%%%%%%%%%%%%%%%%%%%%%%%%%%%%%%%%%%%%%%%%%%%%%%%%%%%%%%%%%%%%
\newsavebox{\polarCirclesFine}
\savebox{\polarCirclesFine}{%
\begin{picture}(6.5,6.5)(-3.25,-3.25)
%\thicklines
\color[gray]{0.8}
% Argument bei circle ist der Durchmesser!
\put(0,0){\circle{1.0}}
\put(0,0){\circle{2.0}}
\put(0,0){\circle{3.0}}
\put(0,0){\circle{4.0}}
\put(0,0){\circle{5.0}}
\put(0,0){\circle{6.0}}
\put(0,0){\circle{7.0}}
\color[gray]{0}
\end{picture}}% Ende \polarCirclesFine
%%%%%%%%%%%%%%%%%%%%%%%%%%%%%%%%%%%%%%%%%%%%%%%%%%%%%%%%%%%%%%%%%%%
\newsavebox{\polarAxis}
\savebox{\polarAxis}{%
\begin{picture}(6.5,6.5)(-3.25,-3.25)
\put(-2.12,-2.12){\line(1,1){4.24}}
\put(2.12, -2.12){\line(-1, 1){4.24}}
\thicklines % damit die "normalen" Striche auch gleich dick sind ...
\put(-3,0){\line(1,0){6.0}}
\put(0,-3){\line(0,1){6.0}}
\end{picture}}% Ende \polarAxis
%%%%%%%%%%%%%%%%%%%%%%%%%%%%%%%%%%%%%%%%%%%%%%%%%%%%%%%%%%%%%%%%%%%
\newsavebox{\polarAxisFine}
\savebox{\polarAxisFine}{%
\begin{picture}(6.5,6.5)(-3.25,-3.25)
\put(-2.47,-2.47){\line(1,1){4.94}}
\put(2.47, -2.47){\line(-1, 1){4.94}}
\thicklines % damit die "normalen" Striche auch gleich dick sind ...
\put(-3.5,0){\line(1,0){7.0}}
\put(0,-3.5){\line(0,1){7.0}}
\end{picture}}% Ende \polarAxisFine
%%%%%%%%%%%%%%%%%%%%%%%%%%%%%%%%%%%%%%%%%%%%%%%%%%%%%%%%%%%%%%%%%%%
\newsavebox{\polarPlot}
\savebox{\polarPlot}{%
\begin{picture}(6.5,6.5)(-3.25,-3.25)
\put(-3.25,-3.25){\usebox{\polarAxis}}
\put(-3.25,-3.25){\usebox{\polarCircles}}
\end{picture}}% Ende \polarPlot
%%%%%%%%%%%%%%%%%%%%%%%%%%%%%%%%%%%%%%%%%%%%%%%%%%%%%%%%%%%%%%%%%%% \newsavebox{\polarPlot}
\newsavebox{\polarPlotFine}
\savebox{\polarPlotFine}{%
\begin{picture}(6.5,6.5)(-3.25,-3.25)
\put(-3.25,-3.25){\usebox{\polarAxisFine}}
\put(-3.25,-3.25){\usebox{\polarCirclesFine}}
\end{picture}}% Ende \polarPlotFine
%%%%%%%%%%%%%%%%%%%%%%%%%%%%%%%%%%%%%%%%%%%%%%%%%%%%%%%%%%%%%%%%%%%

% Beamer theme
\usetheme{CambridgeUS}
\chapter{Folien mit dem Beamer-Package}
Folien werden mit Hilfe des \texttt{Beamer}-Package
erstellt. Dabei stehen die Quellen der Folien parallel zu den Texten und �bungsaufgaben.
Damit wird gew�hrleistet, dass Abbildungen und andere Assets m�glichst  wiederverwendet werden.

Wie schon f�r die Texte gibt es auch hier eine Datei  mit
Angaben, die abh�ngig von der jeweiligen Vorlesung sind. Diese Datei hei�t \datei{header.tex}.
Das folgende Beispiel
stammt aus den Folien f�r die Pr�senzphase des Fachs \emph{Stochastik f�r Informatiker}
im Studiengang IT-Analyst:
\begin{lstlisting}{}
\newcommand{\lectureName}{Stochastik f�r Informatiker}
% Titel der Folie
\title{\lectureName}
\author{Prof. Dr. Manfred Brill}
\institute{Hochschule Kaiserslautern\\
     Fachbereich Informatik und Mikrosystemtechnik}
\date[]{}
% Variablen wie Semester ...
%
\newcommand{\theSemester}{Sommersemester~2018}
% Standardverzeichnis f�r das Basis-Verzeichnis der Bilder
%
\newcommand{\imagePath}{../../images}
%
% Name der Bild-Datei, die auf die Frage-Folie kommen soll
\newcommand{\questionImage}{Misc/Answer_to_Life}
\end{lstlisting}
Die Bezeichnung der Lehrveranstaltung wird definiert und anschlie�end werden die Variablen
f�r die Titelfolie gesetzt.
Anschlie�end definieren wir das aktuelle Semester und die Pfade f�r das Verzeichnis, in dem Bilder f�r die Lehrveranstaltung liegen. Jede Folie enth�lt eine letzte Folie mit einem Bild. Das Bild, das verwendet werden soll wird in der Variable \lstinline$questionImage$
angegeben.

An die Stelle der Datei \texttt{setup.tex} tritt
\datei{slidesheader.tex}, die im Verzeichnis \datei{texmf-local} steht und
unabh�ngig von den Lehrveranstaltungen verwendet werden kann.
Diese Datei wird im Folgenden noch beschrieben.

Bevor wir auf \datei{slidesheader.tex} eingehen werfen wir einen Blick
auf \datei{keywords}. Hier werden Schlagw�rter und andere Informationen f�r
die Eigenschaften des erstellten \pdf{}-Dokuments gespeichert. Das Vorgehen ist
�hnlich wie bei den Texten. Hier ein Beispiel,
wieder aus der gleichen Veranstaltung:

\begin{lstlisting}{}
\subject{Stochastik f�r Informatiker}
\keywords{Wahrscheinlichkeitsrechnung und Statistik,
          Medieninformatik, Hochschule Kaiserslautern}
%
\usepackage{breakurl}
\hypersetup{
pdfauthor = {Manfred Brill},
pdfsubject = {Stochastik f�r Informatiker},
pdftitle = {Stochastik f�r Informatiker},
pdfkeywords = {Stochastik f�r Informatiker, IT-Analyst,
      Hochschule Kaiserslautern},
pdfpagelayout = SinglePage,
pdfpagemode = UseThumbs,
pdfdisplaydoctitle = true
}

\urlstyle{same}
\end{lstlisting}
Die Datei \datei{slidesheader.tex} l�dt �hnlich wie \datei{setup.tex}
einige Packages. Da es bei Folien durchaus vorkommen kann, dass Medien
eingebunden werden wird das Package \lstinline$multimedia$ geladen.
Neben den Folien gibt es Handouts, deshalb wird das Package
\lstinline$pgfpages$ verwendet.
Um Smileys und andere Abbildungen in Folien zu verwenden wird das Package
\lstinline$marvosym$ geladen. Auch f�r die Folien
wird in der aktuellen Version das Paket
\lstinline$xpicture$ geladen. Damit k�nnen alle Grafiken, die
bisher mit \lstinline$eepic$ und \lstinline$dvips$ erstellt wurden auch
mit PDF\LaTeX{} verwendet werden.

Wie f�r die anderen Dokumente werden die Einstellungen
in den Dateien \lstinline$colors.tex$, \lstinline$variablen.tex$ und \lstinline$coordinateSystems.tex$
geladen. Auch das \lstinline$listings$-Paket wird geladen und konfiguriert.
F�r Folien wird die Variable \lstinline$solutions$ definiert, die anschlie�end
in den Folien und auch in \lstinline$slidesheader$ verwendet wird.

Das \lstinline$beamer$-Package bietet verschiedene Styles an, die angepasst werden k�nnen.
Als \emph{theme} wird \lstinline$CambridgeUS$ ausgew�hlt. Die weiteren Anpassungen
finden in der Datei \datei{beamer.tex} statt.
Wichtig ist die Verwendung der Farbe \lstinline$colorDepartment$, die in \datei{colors.tex}
definiert ist. Hier wird die Farbe des Fachbereichs aus dem Logo der Hochschule verwendet. Das Ergebnis
ist in Abbildung \ref{titelfolie} zu sehen.

Folien die nur aus Abbildung und einer �berschrift bestehen kommen h�ufig in Pr�sentationen vor.
Deshalb ist eine Funktion
definiert, mit deren Hilfe man eine solche Folie erstellen kann:
\begin{lstlisting}{}
\newcommand{\imageslide}[3]{%
\begin{frame}{#1}%
  \begin{center}%
   \includegraphics[width = #2]{\imagePath/#3}%
  \end{center}%
\end{frame}}
\end{lstlisting}
Diese Funktion hat drei Argumente. Das erste Argument ist der Titel der Folie, gefolgt
von der Breite, mit der die Abbildung in die Folie eingef�hrt werden soll. Das letzte Argument
ist der Pfad zur Bild-Datei, relativ zum Wert der Variable \lstinline$imagePath$.

\begin{lstlisting}{}
\imageslide{Die Startseite}{6cm}{OLAT/itastochastik_index}
\end{lstlisting}
In Abbildung \ref{bildfolie} ist das Ergebnis dieser Anweisung zu sehen.

\begin{figure}[ht]
\begin{center}
\includegraphics[width=8cm]{\imagePath/imageFolienBeispiel}
\caption{\label{bildfolie}Eine Folie, erzeugt mit \lstinline$imageslide$}
\end{center}
\end{figure}
Am Ende jeder Datei mit Folien wird eine Folie eingef�gt, die die Zuh�rer zu Fragen auffordert. Die H�rer
erkennen so immer, dass das Thema abgeschlossen ist. Daf�r gibt es die Funktion \lstinline$questions$.
Als Bild wird die Datei verwendet, die in \datei{header.tex} in der Variable \lstinline$\questionImage$
definiert ist. Die Funktion \lstinline$questions$ reagiert auf die Variable \lstinline$solutions$.
Die Folie wird nur erzeugt, wenn \lstinline$solutions$ auf \lstinline$TRUE$ gesetzt ist. So kann
die Fragefolie bei Handouts und anderen Dokumenten ausgeblendet werden. F�r die Studierenden
macht es sicher keinen Sinn, in $50$ PDF-Dateien jedes Mal die Fragefolie zu drucken.
Abbildung \ref{fragefolie} zeigt ein Ergebnis der Funktion.

\begin{figure}[ht]
\begin{center}
\includegraphics[width=8cm]{\imagePath/questionsSlide}
\caption{\label{fragefolie}Eine Folie, erzeugt mit \lstinline$questions$}
\end{center}
\end{figure}
Die Breite des eingef�gten Bilds kann als optionales Argument �bergeben werden; der Default-Wert daf�r ist
$5$~cm. Ein Bild mit einer Breite von $8$~cm wird so eingef�gt:
\begin{lstlisting}{}
\questions[8cm]
\end{lstlisting}

Mit den beschriebenen Konfigurationen
sieht der Kopf der \TeX{}-Datei f�r die Folien immer gleich aus:
\begin{lstlisting}{}
\documentclass{beamer}
\input{header}
\input{slidesheader}
%
\subtitle{Organisatorisches}
%
\input{keywords}
% Kommentieren der folgenden Zeilen, je nach Kontext
\setbeameroption{show notes}
\setboolean{solutions}{true}
\pgfpagesuselayout{2 on 1}[a4paper, border shrink=5mm]
%
\end{lstlisting}
Das Thema der Folien wird mit \lstinline$subtitle$ auf die Titel-folie eingef�gt. Die Variable
\lstinline$solutions$ wird gesetzt, damit k�nnen wir Folien aus- und einblenden.
Das \lstinline$beamer$-Package bietet die M�glichkeit, Notizen in die Datei einzuf�gen.
Diese werden als Default ausgeblendet, mit der Zeile
\begin{lstlisting}{}
setbeameroptions{show notes}
\end{lstlisting}
werden sie angezeigt. Die Zeile mit der Funktion \lstinline$pgfpagesuselayout$
wird f�r Handouts eingesetzt, bei denen zwei Folien pro Seite ausgegeben werden.
Abbildung \ref{handouts} zeigt ein solches Beispiel.

\begin{figure}[ht]
\begin{center}
\includegraphics[height=7cm]{\imagePath/handoutsExample}
\caption{\label{handouts}Die Informationen aus unserem Beispiel}
\end{center}
\end{figure}
Insgesamt werden immer drei Versionen des PDF-Dokuments erzeugt. Daf�r gibt es im Verzeichnis mit
den Quelldateien ein Verzeichnis mit dem Namen \datei{fertig} und drei Unterverzeichnisse f�r
Folien, Handouts und Notizen. Alle Folien und alle Notizen liegen in \datei{notes}, die Folien, die
angezeigt werden sollen im Verzeichnis \datei{folien}, und die Handouts, die �ber OLAT
oder andere Seiten an die H�rer verteilt werden liegen im Ordner \datei{handouts}. Die Handouts
werden mit \lstinline$\setboolean{solutions{false}$ und ohne Notizen erzeugt.

\begin{figure}[ht]
\begin{center}
\includegraphics[width=8cm]{\imagePath/beispielfolie}
\caption{\label{titelfolie}Die Titel-Folie zu unserem Beispiel}
\end{center}
\end{figure}
Um in den Notizen eine �berschrift zu formatieren gibt es in \datei{slidesheader}
die Funktion \lstinline$\noteshead$, an die ein Text f�r die �berschrift �bergeben wird.
Die ersten beiden Eintr�ge in jeder Folien-Datei besteht dann aus der Titel-Folie
und einer Notizen-Seite. Auf dieser Notizen-Seite sind Angaben �ber den Inhalt, den
Namen der Datei und �ber die Rangfolge enthalten.
Hier ein Beispiel, das zu der Titel-Folie in Abbildung \ref{titelfolie} f�hrt:
\begin{lstlisting}{}
\begin{frame}
  \titlepage
  \begin{center}
   \includegraphics[width = 3.5cm]
          {\imagePath/statistiklogo}
  \end{center}
\end{frame}

\note{
\noteshead{Information}

\begin{description}
\item[Thema:]  Einf�hrung
\item[Version:] \theSemester{}
\item[Dateiname:] organisatorisches.tex
\item[Reihenfolge:] $1$
\end{description}
}
\end{lstlisting}
Abbildung \ref{infoSlide} zeigt die so erzeugte Notizen-Seite.

\begin{figure}[ht]
\begin{center}
\includegraphics[width=8cm]{\imagePath/infoNote}
\caption{\label{infoSlide}Die Informationen aus unserem Beispiel}
\end{center}
\end{figure}


% Listingspaket
\usepackage[savemem]{listings}
\lstloadlanguages{C++}
\lstset{language=C++}
\lstset{backgroundcolor=\color{lstback}}
\lstset{extendedchars=true}
\lstset{showstringspaces = false}
\lstset{basicstyle = \ttfamily \small}
%% listings mit listings.sty
%
% Funktion f�r die �berschrift in den Notes
\newcommand*{\noteshead}[1]{\textbf{\large #1\normalsize}}
%
% Schritte in einer Aufz�hlung, daf�r einen Z�hler (schritt) und die Umgebung
% schritte definieren.
\newcounter{schritt}
\newenvironment{schritte}%
{\begin{list}%
{Schritt \arabic{schritt}:}%
{\usecounter{schritt}\settowidth{\labelwidth}{Schritt 1:}%
\setlength{\leftmargin}{\labelwidth}\addtolength\leftmargin{\labelsep}%
\parsep0.0ex\partopsep-0.3ex\itemsep2pt\topsep0.0ex}}{\end{list}}
%
% Umgebung f�r die Darstellung eines Algorithmus wie in der 2. Auflage des Mathebuchs
%
\newcommand{\algorithmus}[2]{\vspace{4pt}\fboxsep 1mm \framebox[115mm]%
{\parbox{112mm}{{\bf #1}\vspace{2pt}#2}}\vspace{4pt}}

% Schalter f�r das ein- und ausblenden der L�sungen
\newboolean{solutions}

% Folie mit Bild und Fragen? ...
% Auf den Handouts kommt die Fragefolie nicht vor; die braucht man
% ja nicht jedes Mal auszudrucken.
%
\newcommand*{\questions}[1][5cm]{%
\ifthenelse{\boolean{solutions}}{%
\section{Fragen?}%
\begin{frame}{Unklarheiten? Fragen? Bemerkungen?}%
\begin{center}%
\includegraphics[width=#1]{\imagePath/\questionImage}%
\end{center}%
\end{frame}}{}%
}

% Umgebung f�r eine Folie mit Bild
% Argumente: #1 �berschrift der Folie
%                   #2 Breite des Bilds
%                   #3 Bilddatei (ohne Endung, und mit Defaultordner
%
\newcommand*{\imageslide}[3]{%
\begin{frame}{#1}%
  \begin{center}%
   \includegraphics[width = #2]{\imagePath/#3}%
  \end{center}%
\end{frame}}

%
\subtitle{Organisatorisches}
%
\input{keywords}
% Kommentieren der folgenden Zeilen, je nach Kontext
\setbeameroption{show notes}
\setboolean{solutions}{true}
\pgfpagesuselayout{2 on 1}[a4paper, border shrink=5mm]
%
\end{lstlisting}
Das Thema der Folien wird mit \lstinline$subtitle$ auf die Titel-folie eingef�gt. Die Variable
\lstinline$solutions$ wird gesetzt, damit k�nnen wir Folien aus- und einblenden.
Das \lstinline$beamer$-Package bietet die M�glichkeit, Notizen in die Datei einzuf�gen.
Diese werden als Default ausgeblendet, mit der Zeile
\begin{lstlisting}{}
setbeameroptions{show notes}
\end{lstlisting}
werden sie angezeigt. Die Zeile mit der Funktion \lstinline$pgfpagesuselayout$
wird f�r Handouts eingesetzt, bei denen zwei Folien pro Seite ausgegeben werden.
Abbildung \ref{handouts} zeigt ein solches Beispiel.

\begin{figure}[ht]
\begin{center}
\includegraphics[height=7cm]{\imagePath/handoutsExample}
\caption{\label{handouts}Die Informationen aus unserem Beispiel}
\end{center}
\end{figure}
Insgesamt werden immer drei Versionen des PDF-Dokuments erzeugt. Daf�r gibt es im Verzeichnis mit
den Quelldateien ein Verzeichnis mit dem Namen \datei{fertig} und drei Unterverzeichnisse f�r
Folien, Handouts und Notizen. Alle Folien und alle Notizen liegen in \datei{notes}, die Folien, die
angezeigt werden sollen im Verzeichnis \datei{folien}, und die Handouts, die �ber OLAT
oder andere Seiten an die H�rer verteilt werden liegen im Ordner \datei{handouts}. Die Handouts
werden mit \lstinline$\setboolean{solutions{false}$ und ohne Notizen erzeugt.

\begin{figure}[ht]
\begin{center}
\includegraphics[width=8cm]{\imagePath/beispielfolie}
\caption{\label{titelfolie}Die Titel-Folie zu unserem Beispiel}
\end{center}
\end{figure}
Um in den Notizen eine �berschrift zu formatieren gibt es in \datei{slidesheader}
die Funktion \lstinline$\noteshead$, an die ein Text f�r die �berschrift �bergeben wird.
Die ersten beiden Eintr�ge in jeder Folien-Datei besteht dann aus der Titel-Folie
und einer Notizen-Seite. Auf dieser Notizen-Seite sind Angaben �ber den Inhalt, den
Namen der Datei und �ber die Rangfolge enthalten.
Hier ein Beispiel, das zu der Titel-Folie in Abbildung \ref{titelfolie} f�hrt:
\begin{lstlisting}{}
\begin{frame}
  \titlepage
  \begin{center}
   \includegraphics[width = 3.5cm]
          {\imagePath/statistiklogo}
  \end{center}
\end{frame}

\note{
\noteshead{Information}

\begin{description}
\item[Thema:]  Einf�hrung
\item[Version:] \theSemester{}
\item[Dateiname:] organisatorisches.tex
\item[Reihenfolge:] $1$
\end{description}
}
\end{lstlisting}
Abbildung \ref{infoSlide} zeigt die so erzeugte Notizen-Seite.

\begin{figure}[ht]
\begin{center}
\includegraphics[width=8cm]{\imagePath/infoNote}
\caption{\label{infoSlide}Die Informationen aus unserem Beispiel}
\end{center}
\end{figure}


% Listingspaket
\usepackage[savemem]{listings}
\lstloadlanguages{C++}
\lstset{language=C++}
\lstset{backgroundcolor=\color{lstback}}
\lstset{extendedchars=true}
\lstset{showstringspaces = false}
\lstset{basicstyle = \ttfamily \small}
%% listings mit listings.sty
%
% Funktion f�r die �berschrift in den Notes
\newcommand*{\noteshead}[1]{\textbf{\large #1\normalsize}}
%
% Schritte in einer Aufz�hlung, daf�r einen Z�hler (schritt) und die Umgebung
% schritte definieren.
\newcounter{schritt}
\newenvironment{schritte}%
{\begin{list}%
{Schritt \arabic{schritt}:}%
{\usecounter{schritt}\settowidth{\labelwidth}{Schritt 1:}%
\setlength{\leftmargin}{\labelwidth}\addtolength\leftmargin{\labelsep}%
\parsep0.0ex\partopsep-0.3ex\itemsep2pt\topsep0.0ex}}{\end{list}}
%
% Umgebung f�r die Darstellung eines Algorithmus wie in der 2. Auflage des Mathebuchs
%
\newcommand{\algorithmus}[2]{\vspace{4pt}\fboxsep 1mm \framebox[115mm]%
{\parbox{112mm}{{\bf #1}\vspace{2pt}#2}}\vspace{4pt}}

% Schalter f�r das ein- und ausblenden der L�sungen
\newboolean{solutions}

% Folie mit Bild und Fragen? ...
% Auf den Handouts kommt die Fragefolie nicht vor; die braucht man
% ja nicht jedes Mal auszudrucken.
%
\newcommand*{\questions}[1][5cm]{%
\ifthenelse{\boolean{solutions}}{%
\section{Fragen?}%
\begin{frame}{Unklarheiten? Fragen? Bemerkungen?}%
\begin{center}%
\includegraphics[width=#1]{\imagePath/\questionImage}%
\end{center}%
\end{frame}}{}%
}

% Umgebung f�r eine Folie mit Bild
% Argumente: #1 �berschrift der Folie
%                   #2 Breite des Bilds
%                   #3 Bilddatei (ohne Endung, und mit Defaultordner
%
\newcommand*{\imageslide}[3]{%
\begin{frame}{#1}%
  \begin{center}%
   \includegraphics[width = #2]{\imagePath/#3}%
  \end{center}%
\end{frame}}

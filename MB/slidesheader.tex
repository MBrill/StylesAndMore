% --------------------------------------------------------------------------------
%    Farben, Optionen, Packages und Kommandos f�r Folien
% --------------------------------------------------------------------------------
\typeout{Beamer-Folien an der Hochschule Kaiserslautern}
\typeout{     (C) Manfred Brill}
\typeout{     Version 1.4 Oktober 2023}
\typeout{     PDF-Support f�r LaTeX-Grafiken}
\typeout{     Bei Verwendung von dvi -> ps -> pdf mb.sty verwenden!}
\usepackage[german]{babel}
\usepackage[ansinew]{inputenc}
\usepackage[T1]{fontenc}
\usepackage{times}
\usepackage{amsmath}
\usepackage{amsfonts}
\usepackage{amssymb}
\usepackage{amscd}
\usepackage{mbmath}
\usepackage{color}
\usepackage{graphicx}
\usepackage{xpicture}
\usepackage{float}
% Hack f�r Warnings in KomaScript
\usepackage{scrhack} % suppresses \foat@addtolist warning
% Paket f�r Hervorhebungen, Durchstreichen, ...
\usepackage[normalem]{ulem}
% Coole Symbole, wie smileys, B�gelzeichen, ...
\usepackage{marvosym}
% Paket f�r Euro-Symbol. Danach gibt es die Funktion \euro{} f�r das
% Symbol; und \EUR{1,10} f�r einen Betrag.
% F�r den Text 1,10 � verwendet man \EUR{1,10}. Die Option left f�hrt
% dazu, dass der Text als � 1,10 ausgegeben wird.
\usepackage[right]{eurosym}
% ifthen f�r Ein- und Ausblenden der L�sungen.
\usepackage{ifthen}
% Mehr Kontrollen �ber Tabellen
\usepackage{array}
% Paket f�r Filme
\usepackage{multimedia}
\usepackage{pgfpages}
% Keine subsections im Inhaltsverzeichnis
\setcounter{tocdepth}{1}
% --- Dateien aus texmf-local/tex/latex/MB
% # colors
% # variablen
% # coordinateSystems
% # beamer
% Farben
% ---------------------------------------------------------------------------------------
% Farben, f�r Folien und andere Dinge.
% Version vom Juli 2015 mit neuer Fachbereichsfarbe
%
% Letzte �nderung: 1.9.2015
%
% Um diese Datei zu verwenden muss Sie in texmf-local/MB kopiert werden
% und TeX muss aktualisiert werden!
% ---------------------------------------------------------------------------------------
\definecolor{light}{gray}{.75}
\definecolor{LightGray}{rgb}{0.24,0.24,0.24}
%   Farben rot gr�n blau f�r Koordinatenachsen
\definecolor{wred}{rgb}{0.8, 0.0, 0.0}
\definecolor{wgreen}{rgb}{0.0, 0.8, 0.0}
\definecolor{wblue}{rgb}{0.0, 0.0, 0.8}
\definecolor{wyellow}{rgb}{1.0, 1.0, 0.0}
\definecolor{wcyan}{rgb}{0.0, 1.0, 1.0}
% Fachbereichsfarbe neu
\definecolor{colorDepartment}{rgb}{0.16, 0.71, 0.86}
% Listings
\definecolor{lstback}{gray}{0.85}
%

% Variablen f�r Namen, Software, ...
% -------------------------------------------------------------------
% variablen.tex
% Variablen f�r Werkzeuge, Begriffe, Firmen
% Stand: Juni 2017
% Liegt zentral in texmf-local/tex/latex/MB
% -------------------------------------------------------------------
% Programmiersprachen
\newcommand{\java}{\texttt{Java}}
\newcommand{\cpp}{\texttt{C++}}
\newcommand{\cl}{\texttt{C}}
\newcommand{\csharp}{\texttt{C\#}}
\newcommand{\php}{\texttt{PHP}}
\newcommand{\py}{\texttt{Python}}
\newcommand{\perl}{\texttt{Perl}}
\newcommand{\fc}{\texttt{Fortran}}
\newcommand{\obc}{\texttt{Objective C}}
\newcommand{\js}{\texttt{JavaScript}}
\newcommand{\omp}{\texttt{OpenMP}}
\newcommand{\mpi}{\texttt{MPI}}
\newcommand{\ocl}{\texttt{OpenCL}}
\newcommand{\cuda}{\texttt{CUDA}}
% Datei-Formate
\newcommand{\html}{\texttt{HTML}}
\newcommand{\xml}{\texttt{XML}}
\newcommand{\json}{\texttt{JSON}}
\newcommand{\rtf}{\texttt{RTF}}
\newcommand{\pdf}{\texttt{PDF}}
%
\newcommand{\sccs}{\texttt{SCCS}}
\newcommand{\rcs}{\texttt{RCS}}
\newcommand{\cvs}{\texttt{CVS}}
\newcommand{\svn}{\texttt{SVN}}
\newcommand{\subversion}{\texttt{Subversion}}
\newcommand{\git}{\texttt{Git}}
\newcommand{\github}{\texttt{GitHub}}
\newcommand{\githubDesktop}{\texttt{GitHub Desktop}}
\newcommand{\gitkraken}{\texttt{GitKraken}}
\newcommand{\gittortoise}{\texttt{TortoiseGit}}
\newcommand{\gittree}{\texttt{Sourcetree}}
\newcommand{\bitbucket}{\texttt{BitBucket}}
\newcommand{\mercurial}{\texttt{Mercurial}}
\newcommand{\pf}{\texttt{Perforce}}
\newcommand{\pfv}{\texttt{P4V}}
\newcommand{\pfc}{\texttt{p4}}
\newcommand{\pfmerge}{\texttt{P4Merge}}
\newcommand{\pfsand}{\texttt{P4.Sandbox}}

\newcommand{\vi}{\texttt{vi}}
\newcommand{\emacs}{\texttt{Emacs}}
%
\newcommand{\javadoc}{\texttt{Javadoc}}
\newcommand{\xmldoc}{\texttt{XMLdoc}}
\newcommand{\sandcastle}{\texttt{Sandcastle}}
\newcommand{\doxy}{\texttt{Doxygen}}
\newcommand{\doxyVersion}{\texttt{$1.8.7$}}
\newcommand{\doxywizard}{\texttt{Doxywizard}}
\newcommand{\graphviz}{\texttt{Graphviz}}
%
\newcommand{\logj}{\texttt{Log4j}}
\newcommand{\slfj}{\texttt{SLF4J}}
\newcommand{\logback}{\texttt{LOGBack}}
\newcommand{\jul}{\texttt{Java Logging API}}
\newcommand{\commonsLog}{\texttt{Apache Commons LoggingJ}}
\newcommand{\boostlog}{\texttt{Boost.Log}}
\newcommand{\glog}{\texttt{Google Logging Library}}
%
\newcommand{\junit}{\texttt{JUnit}}
\newcommand{\xunit}{\texttt{xUnit}}
\newcommand{\nunit}{\texttt{NUnit}}
\newcommand{\xunitnet}{\texttt{xUnit.net}}
\newcommand{\ctest}{\texttt{CTest}}
%
\newcommand{\ssh}{\texttt{ssh}}
\newcommand{\sftp}{\texttt{sftp}}
\newcommand{\make}{\texttt{make}}
\newcommand{\ant}{\texttt{Ant}}
\newcommand{\maven}{\texttt{Maven}}
\newcommand{\gradle}{\texttt{Gradle}}
\newcommand{\msbuild}{\texttt{MSBuild}}
\newcommand{\mstest}{\texttt{MSTest}}
\newcommand{\cmake}{\texttt{CMake}}
\newcommand{\jenkins}{\texttt{Jenkins}}
\newcommand{\ccontrol}{\texttt{cruisecontrol}}
\newcommand{\cdash}{\texttt{CDash}}
%
\newcommand{\eclipse}{\texttt{Eclipse}}
\newcommand{\vs}{\texttt{Visual Studio}}
\newcommand{\mono}{\texttt{Mono}}
\newcommand{\monodev}{\texttt{MonoDevelop}}
%
\newcommand{\openO}{\texttt{OpenOffice}}
\newcommand{\Qt}{\texttt{Qt}}
\newcommand{\firefox}{\texttt{Mozilla Firefox}}
\newcommand{\kde}{\texttt{KDE}}
%
\newcommand{\ms}{\texttt{Microsoft}}
\newcommand{\net}{\texttt{.NET}}
\newcommand{\sun}{\texttt{SUN}}
\newcommand{\ora}{\texttt{Oracle}}
\newcommand{\apache}{\texttt{Apache}}
\newcommand{\windows}{\texttt{Windows}}
\newcommand{\cyg}{\texttt{Cygwin}}
\newcommand{\linux}{\texttt{Linux}}
\newcommand{\unix}{\texttt{Unix}}
\newcommand{\osx}{\texttt{MacOS X}}
\newcommand{\android}{\texttt{Android}}
% VR Software
\newcommand{\unity}{\texttt{Unity}}
\newcommand{\verUnity}{\texttt{$2018.2$}}
\newcommand{\unreal}{\texttt{Unreal}}
\newcommand{\verUnreal}{\texttt{4}}
\newcommand{\juggler}{\texttt{VRJuggler}}
\newcommand{\cavelib}{\texttt{CAVELib}}
\newcommand{\gadgeteer}{\texttt{Gadgeteer}}
\newcommand{\vrpn}{\texttt{VRPN}}
\newcommand{\mvr}{\texttt{MiddleVR}}
\newcommand{\verMvr}{\texttt{$1.7.0.7$}}
% Computergrafik und Werkzeuge
\newcommand{\direct}{\texttt{Direct3D}}
\newcommand{\gl}{\texttt{OpenGL}}
\newcommand{\para}{\texttt{ParaView}}
\newcommand{\vtk}{\texttt{VTK}}
% R und anderes zur Datenanalyse
% \R ist schon definiert (Mathematik!)
\newcommand{\Rsoft}{\texttt{R}}
\newcommand{\Rstudio}{\texttt{RStudio}}
\newcommand{\verR}{\texttt{$3.3.3$}}
\newcommand{\mondrian}{\texttt{Mondrian}}
\newcommand{\tableau}{\texttt{Tableau}}
% LaTeX
\newcommand{\jabref}{\texttt{JabRef}}
\newcommand{\bibtex}{\textsc{Bib}\negthinspace\TeX}

% Beamer theme
\usetheme{CambridgeUS}
%
% Beamer Thema f�r Vorlesungsfolien
% Letzte �nderung: 1.9.2015
%
% Um diese Datei zu verwenden muss Sie in texmf-local/MB kopiert werden
% und TeX muss aktualisiert werden!
%
% statt red eine definierte Farbe ...
\usecolortheme[named=colorDepartment]{structure}
\makeatletter
\setbeamercolor{departmentColor}{fg=black, bg=colorDepartment}
\setbeamercolor{headline}{fg=black}
\setbeamercolor{title}{bg=gray!30!white, fg=black}
%
\setbeamertemplate{navigation symbols}[vertical]
\setbeamertemplate{navigation symbols}{\insertslidenavigationsymbol}
%
\setbeamercolor{frametitle}{fg=white!10!black, bg=white!80!black}
%
\addtobeamertemplate{theorem begin}{%
  \setbeamercolor{block title}{fg=black, bg=gray!30!white}%
  %\setbeamercolor{block body}{fg=red,bg=yellow}%
}{}
\addtobeamertemplate{proof begin}{%
  \setbeamercolor{block title}{fg=black, bg=gray!30!white}%
}{}
\addtobeamertemplate{block begin}{%
  \setbeamercolor{block title}{fg=black, bg=gray!30!white}%
}{}
\makeatother
%
%
\defbeamertemplate*{footline}{vislab infolines}
{
  \leavevmode%
  \hbox{%
  \begin{beamercolorbox}[wd=.333333\paperwidth,ht=2.25ex,dp=1ex,left]{departmentColor}%
    \usebeamerfont{author in head/foot}\hspace*{1ex}\color{black} \lectureName{}
  \end{beamercolorbox}%
  \begin{beamercolorbox}[wd=.333333\paperwidth,ht=2.25ex,dp=1ex,center]{title in head/foot}%
    \usebeamerfont{title in head/foot}\color{black}\insertshortsubtitle
  \end{beamercolorbox}%
  \begin{beamercolorbox}[wd=.333333\paperwidth,ht=2.25ex,dp=1ex,right]{date in head/foot}%
    \usebeamerfont{date in head/foot}\color{black}\insertframenumber{} \hspace*{1ex}
  \end{beamercolorbox}}%
  \vskip0pt%
}

\defbeamertemplate*{headline}{vislab infolines}
{
  \leavevmode%
  \hbox{%
  \begin{beamercolorbox}[wd=.5\paperwidth,ht=2.25ex,dp=1ex,right]{departmentColor}%{section in head/foot}%
    \usebeamerfont{section in head/foot}\color{black}\insertsectionhead\hspace*{2ex}
  \end{beamercolorbox}%
  \begin{beamercolorbox}[wd=.5\paperwidth,ht=2.25ex,dp=1ex,right]{subsection in head/foot}%
    \usebeamerfont{subsection in head/foot}\color{black}\insertsubsectionhead\hspace*{2ex}
  \end{beamercolorbox}}%
  \vskip0pt%
}
%
\setbeamertemplate{footline}[vislab infolines]
\setbeamertemplate{headline}[vislab infolines]
\setbeamertemplate{note page}[plain] 

% Listingspaket
\usepackage[savemem]{listings}
\lstloadlanguages{C++}
\lstset{language=C++}
\lstset{backgroundcolor=\color{lstback}}
\lstset{extendedchars=true}
\lstset{showstringspaces = false}
\lstset{basicstyle = \ttfamily \small}
%% listings mit listings.sty
%
% Funktion f�r die �berschrift in den Notes
\newcommand*{\noteshead}[1]{\textbf{\large #1\normalsize}}
%
% Schritte in einer Aufz�hlung, daf�r einen Z�hler (schritt) und die Umgebung
% schritte definieren.
\newcounter{schritt}
\newenvironment{schritte}%
{\begin{list}%
{Schritt \arabic{schritt}:}%
{\usecounter{schritt}\settowidth{\labelwidth}{Schritt 1:}%
\setlength{\leftmargin}{\labelwidth}\addtolength\leftmargin{\labelsep}%
\parsep0.0ex\partopsep-0.3ex\itemsep2pt\topsep0.0ex}}{\end{list}}
%
% Umgebung f�r die Darstellung eines Algorithmus wie in der 2. Auflage des Mathebuchs
%
\newcommand{\algorithmus}[2]{\vspace{4pt}\fboxsep 1mm \framebox[145mm]%
{\parbox{142mm}{{\bf #1}\vspace{2pt}#2}}\vspace{4pt}}

% Schalter f�r das ein- und ausblenden der L�sungen
\newboolean{solutions}

% Folie mit Bild und Header
%
% Auf den Handouts kommt die Fragefolie nicht vor; die braucht man
% ja nicht jedes Mal auszudrucken.
%
\newcommand*{\questions}[1][5cm]{%
\ifthenelse{\boolean{solutions}}{%
\section{Fragen?}%
\begin{frame}{Unklarheiten? Fragen? Bemerkungen?}%
\begin{center}%
\includegraphics[width=#1]{\imagePath/\questionImage}%
\end{center}%
\end{frame}}{}%
}

% Folie mit Bild und Fragen mit englischem Text
%
\newcommand*{\questionsEnglish}[1][5cm]{%
\ifthenelse{\boolean{solutions}}{%
\section{Questions?}%
\begin{frame}{Questions? Remarks?}%
\begin{center}%
\includegraphics[width=#1]{\imagePath/\questionImage}%
\end{center}%
\end{frame}}{}%
}

% Umgebung f�r eine Folie mit Bild
% Argumente: #1 �berschrift der Folie
%                   #2 Breite des Bilds
%                   #3 Bilddatei (ohne Endung, und mit Defaultordner
%
\newcommand*{\imageslide}[3]{%
\begin{frame}{#1}%
  \begin{center}%
   \includegraphics[width = #2]{\imagePath/#3}%
  \end{center}%
\end{frame}}

% Darstellung von Punkten mit xpicture mit Hilfe von \pointmark
\renewcommand{\pointmark}{$\bullet$}
% Funktion, der die Koordinaten �bergeben werden und einen Punkt ausgibt
\newcommand{\drawPoint}[2]{\Put[c](#1, #2){\pointmark}}

